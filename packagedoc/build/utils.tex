%
% Generated at 26.08.2022 - 15:39:48 by QConnectBase
%

\hypertarget{qconnectbase-utils-class-singleton-77}{%
\section{Class: Singleton}\label{qconnectbase-utils-class-singleton-77}}

\begin{Shaded}
\begin{Highlighting}[]
\NormalTok{QConnectBase.utils}
\end{Highlighting}
\end{Shaded}

Class to implement Singleton Design Pattern. This class is used to
derive the TTFisClientReal as only a single instance of this class is
allowed.

Disabled pyLint Messages: R0903: Too few public methods (\%s/\%s) Used
when class has too few public methods, so be sure it\textquotesingle s
really worth it.

\begin{quote}
This base class implements the Singleton Design Pattern required for the
TTFisClientReal. Adding further methods does not make sense.
\end{quote}

\hypertarget{qconnectbase-utils-class-dicttoclass-78}{%
\section{Class: DictToClass}\label{qconnectbase-utils-class-dicttoclass-78}}

\begin{Shaded}
\begin{Highlighting}[]
\NormalTok{QConnectBase.utils}
\end{Highlighting}
\end{Shaded}

Class for converting dictionary to class object. Method: validate
-\/-\/-\/-\/-\/-\/-\/-\/-\/-\/-\/-\/-\/-\/-\/-

\hypertarget{qconnectbase-utils-class-utils-79}{%
\section{Class: Utils}\label{qconnectbase-utils-class-utils-79}}

\begin{Shaded}
\begin{Highlighting}[]
\NormalTok{QConnectBase.utils}
\end{Highlighting}
\end{Shaded}

Class to implement utilities for supporting development. Method:
get\_all\_descendant\_classes
-\/-\/-\/-\/-\/-\/-\/-\/-\/-\/-\/-\/-\/-\/-\/-\/-\/-\/-\/-\/-\/-\/-\/-\/-\/-\/-\/-\/-\/-\/-\/-\/-\/-

Get all descendant classes of a class

\begin{description}
\item[\textbf{Arguments:}]
cls: Input class for finding descendants.
\item[\textbf{Returns:}]
Array of descendant classes.
\end{description}

\hypertarget{qconnectbase-utils-method-get_all_sub_classes-80}{%
\subsection{Method:
get\_all\_sub\_classes}\label{qconnectbase-utils-method-get_all_sub_classes-80}}

Get all children classes of a class

\textbf{Arguments:}

\begin{itemize}
\item
  \texttt{cls}

  / \emph{Condition}: required / \emph{Type}: class /

  Input class for finding children.
\end{itemize}

\begin{description}
\item[\textbf{Returns:}]
Array of children classes.
\end{description}

\hypertarget{qconnectbase-utils-method-is_valid_host-81}{%
\subsection{Method: is\_valid\_host}\label{qconnectbase-utils-method-is_valid_host-81}}

\hypertarget{qconnectbase-utils-method-execute_command-82}{%
\subsection{Method: execute\_command}\label{qconnectbase-utils-method-execute_command-82}}

\hypertarget{qconnectbase-utils-method-kill_process-83}{%
\subsection{Method: kill\_process}\label{qconnectbase-utils-method-kill_process-83}}

\hypertarget{qconnectbase-utils-method-caller_name-84}{%
\subsection{Method: caller\_name}\label{qconnectbase-utils-method-caller_name-84}}

Get a name of a caller in the format module.class.method

\textbf{Arguments:}

\begin{itemize}
\item
  \texttt{skip}

  / \emph{Condition}: required / \emph{Type}: int /

  \begin{description}
  \item[Specifies how many levels of stack to skip while getting caller]
  name. skip=1 means "who calls me", skip=2 "who calls my caller" etc.
  \end{description}
\end{itemize}

\begin{description}
\item[\textbf{Returns:}]
An empty string is returned if skipped levels exceed stack height
\end{description}

\hypertarget{qconnectbase-utils-method-load_library-85}{%
\subsection{Method: load\_library}\label{qconnectbase-utils-method-load_library-85}}

Load native library depend on the calling convention.

\begin{description}
\item[\textbf{Arguments:}]
path: library path. is\_stdcall: determine if the
library\textquotesingle s calling convention is stdcall or cdecl.
\item[\textbf{Returns:}]
Loaded library object.
\end{description}

\hypertarget{qconnectbase-utils-method-is_ascii_or_unicode-86}{%
\subsection{Method:
is\_ascii\_or\_unicode}\label{qconnectbase-utils-method-is_ascii_or_unicode-86}}

Check if the string is ascii or unicode

\begin{description}
\item[\textbf{Arguments:}]
str\_check: string for checking codecs: encoding type list
\item[\textbf{Returns:}]
True : if checked string is ascii or unicode False : if checked string
is not ascii or unicode
\end{description}

\hypertarget{qconnectbase-utils-class-job-87}{%
\section{Class: Job}\label{qconnectbase-utils-class-job-87}}

\begin{Shaded}
\begin{Highlighting}[]
\NormalTok{QConnectBase.utils}
\end{Highlighting}
\end{Shaded}

\hypertarget{qconnectbase-utils-method-stop-88}{%
\subsection{Method: stop}\label{qconnectbase-utils-method-stop-88}}

\hypertarget{qconnectbase-utils-method-run-89}{%
\subsection{Method: run}\label{qconnectbase-utils-method-run-89}}

\hypertarget{qconnectbase-utils-class-resulttype-90}{%
\section{Class: ResultType}\label{qconnectbase-utils-class-resulttype-90}}

\begin{Shaded}
\begin{Highlighting}[]
\NormalTok{QConnectBase.utils}
\end{Highlighting}
\end{Shaded}

Result Types. Class: ResponseMessage ======================

\begin{Shaded}
\begin{Highlighting}[]
\NormalTok{QConnectBase.utils}
\end{Highlighting}
\end{Shaded}

Response message class

\hypertarget{qconnectbase-utils-method-get_json-91}{%
\subsection{Method: get\_json}\label{qconnectbase-utils-method-get_json-91}}

Convert response message to json

\begin{description}
\item[\textbf{Returns:}]
Response message in json format
\end{description}

\hypertarget{qconnectbase-utils-method-get_data-92}{%
\subsection{Method: get\_data}\label{qconnectbase-utils-method-get_data-92}}

Get string data result

\begin{description}
\item[\textbf{Returns:}]
String result
\end{description}

\hypertarget{qconnectbase-utils-method-create_from_string-93}{%
\subsection{Method:
create\_from\_string}\label{qconnectbase-utils-method-create_from_string-93}}
