%% Generated by Sphinx.
\def\sphinxdocclass{report}
\documentclass[letterpaper,10pt,english]{sphinxmanual}
\ifdefined\pdfpxdimen
   \let\sphinxpxdimen\pdfpxdimen\else\newdimen\sphinxpxdimen
\fi \sphinxpxdimen=.75bp\relax
\ifdefined\pdfimageresolution
    \pdfimageresolution= \numexpr \dimexpr1in\relax/\sphinxpxdimen\relax
\fi
%% let collapsible pdf bookmarks panel have high depth per default
\PassOptionsToPackage{bookmarksdepth=5}{hyperref}

\PassOptionsToPackage{warn}{textcomp}
\usepackage[utf8]{inputenc}
\ifdefined\DeclareUnicodeCharacter
% support both utf8 and utf8x syntaxes
  \ifdefined\DeclareUnicodeCharacterAsOptional
    \def\sphinxDUC#1{\DeclareUnicodeCharacter{"#1}}
  \else
    \let\sphinxDUC\DeclareUnicodeCharacter
  \fi
  \sphinxDUC{00A0}{\nobreakspace}
  \sphinxDUC{2500}{\sphinxunichar{2500}}
  \sphinxDUC{2502}{\sphinxunichar{2502}}
  \sphinxDUC{2514}{\sphinxunichar{2514}}
  \sphinxDUC{251C}{\sphinxunichar{251C}}
  \sphinxDUC{2572}{\textbackslash}
\fi
\usepackage{cmap}
\usepackage[T1]{fontenc}
\usepackage{amsmath,amssymb,amstext}
\usepackage{babel}



\usepackage{tgtermes}
\usepackage{tgheros}
\renewcommand{\ttdefault}{txtt}



\usepackage[Bjarne]{fncychap}
\usepackage{sphinx}

\fvset{fontsize=auto}
\usepackage{geometry}


% Include hyperref last.
\usepackage{hyperref}
% Fix anchor placement for figures with captions.
\usepackage{hypcap}% it must be loaded after hyperref.
% Set up styles of URL: it should be placed after hyperref.
\urlstyle{same}

\addto\captionsenglish{\renewcommand{\contentsname}{Contents:}}

\usepackage{sphinxmessages}
\setcounter{tocdepth}{3}
\setcounter{secnumdepth}{3}


\title{The QConnection Base Library}
\date{Nov 30, 2021}
\release{}
\author{Nguyen Huynh Tri Cuong (RBVH/ECM1)}
\newcommand{\sphinxlogo}{\vbox{}}
\renewcommand{\releasename}{}
\makeindex
\begin{document}

\ifdefined\shorthandoff
  \ifnum\catcode`\=\string=\active\shorthandoff{=}\fi
  \ifnum\catcode`\"=\active\shorthandoff{"}\fi
\fi

\pagestyle{empty}
\sphinxmaketitle
\pagestyle{plain}
\sphinxtableofcontents
\pagestyle{normal}
\phantomsection\label{\detokenize{index::doc}}




\sphinxAtStartPar
{\hyperref[\detokenize{index:license}]{\emph{\sphinxincludegraphics{{D:\Project\robot\ROBFW-229\robotframework-qconnect-base\doc\_build\doctrees\images\69d3e9dd766a3c0be6304e090529292e75d3a979\bioslv4-badge}.svg}}}}

\sphinxAtStartPar
QConnectBaseLibrary is a connection testing library for \sphinxhref{https://robotframework.org}{Robot
Framework}. Library will be supported to
downloaded from PyPI soon. It provides a mechanism to handle trace log
continously receiving from a connection (such as Raw TCP, SSH, Serial,
etc.) besides sending data back to the other side. It’s especially
efficient for monitoring the overflood response trace log from an
asynchronous trace systems. It is supporting Python 3.7+ and
RobotFramework 3.2+.


\chapter{Table of Contents}
\label{\detokenize{index:table-of-contents}}\begin{itemize}
\item {} 
\sphinxAtStartPar
{\hyperref[\detokenize{index:getting-started}]{\emph{Getting Started}}}

\item {} 
\sphinxAtStartPar
{\hyperref[\detokenize{index:building-and-testing}]{\emph{Usage}}}

\item {} 
\sphinxAtStartPar
{\hyperref[\detokenize{index:example}]{\emph{Example}}}

\item {} 
\sphinxAtStartPar
{\hyperref[\detokenize{index:contribution-guidelines}]{\emph{Contribution Guidelines}}}

\item {} 
\sphinxAtStartPar
{\hyperref[\detokenize{index:configure-Git-and-correct-EOL-handling}]{\emph{Configure Git and correct EOL
handling}}}

\item {} 
\sphinxAtStartPar
{\hyperref[\detokenize{index:documentation}]{\emph{Sourcecode Documentation}}}

\item {} 
\sphinxAtStartPar
{\hyperref[\detokenize{index:feedback}]{\emph{Feedback}}}

\item {} 
\sphinxAtStartPar
{\hyperref[\detokenize{index:about}]{\emph{About}}}
\begin{itemize}
\item {} 
\sphinxAtStartPar
{\hyperref[\detokenize{index:maintainers}]{\emph{Maintainers}}}

\item {} 
\sphinxAtStartPar
{\hyperref[\detokenize{index:contributors}]{\emph{Contributors}}}

\item {} 
\sphinxAtStartPar
{\hyperref[\detokenize{index:3rd-party-licenses}]{\emph{3rd Party Licenses}}}

\item {} 
\sphinxAtStartPar
{\hyperref[\detokenize{index:used-encryption}]{\emph{Used Encryption}}}

\item {} 
\sphinxAtStartPar
{\hyperref[\detokenize{index:license}]{\emph{License}}}

\end{itemize}

\end{itemize}


\chapter{Getting Started}
\label{\detokenize{index:getting-started}}
\sphinxAtStartPar
We have a plan to publish all the sourcecode as OSS in the near future
so that you can downloaded from PyPI. For the current period, you can
checkout all
\sphinxhref{https://sourcecode.socialcoding.bosch.com/projects/ROBFW/repos/robotframework-qconnect-base/browse}{QConnectBaseLibrary}
sourcecode from the Bosch Internal Open Source Repositories.

\sphinxAtStartPar
After checking out the source completely, you can install by running
below command inside \sphinxstylestrong{robotframework\sphinxhyphen{}qconnect\sphinxhyphen{}base} directory.

\begin{sphinxVerbatim}[commandchars=\\\{\}]
\PYG{n}{python} \PYG{n}{setup}\PYG{o}{.}\PYG{n}{py} \PYG{n}{install}
\end{sphinxVerbatim}


\chapter{Usage}
\label{\detokenize{index:usage}}
\sphinxAtStartPar
QConnectBaseLibrary support following keywords for testing connection in RobotFramework.


\section{\sphinxstylestrong{connect}}
\label{\detokenize{index:connect}}\begin{quote}

\sphinxAtStartPar
\sphinxstylestrong{Use for establishing a connection.}

\sphinxAtStartPar
\sphinxstylestrong{Syntax}:
\begin{quote}

\sphinxAtStartPar
\sphinxstylestrong{connect} \sphinxcode{\sphinxupquote{{[}conn\_name{]}   {[}conn\_type{]}   {[}conn\_mode{]}   {[}conn\_conf{]}}}
\sphinxstyleemphasis{(All parameters are required to be in order)}

\sphinxAtStartPar
or

\sphinxAtStartPar
\sphinxstylestrong{connect}
\sphinxcode{\sphinxupquote{conn\_name={[}conn\_name{]}   conn\_type={[}conn\_type{]}   conn\_mode={[}conn\_mode{]}   conn\_conf={[}conn\_conf{]}}}
\sphinxstyleemphasis{(All parameters are assigned by name)}
\end{quote}

\sphinxAtStartPar
\sphinxstylestrong{Arguments}:
\begin{quote}

\sphinxAtStartPar
\sphinxstylestrong{conn\_name}: Name of the connection.

\sphinxAtStartPar
\sphinxstylestrong{conn\_type}: Type of the connection. QConnectBaseLibrary has supported below connection types:
\begin{itemize}
\item {} 
\sphinxAtStartPar
\sphinxstylestrong{TCPIPClient}: Create a Raw TCPIP connection to TCP Server.

\item {} 
\sphinxAtStartPar
\sphinxstylestrong{SSHClient}: Create a client connection to a SSH server.

\item {} 
\sphinxAtStartPar
\sphinxstylestrong{SerialClient}: Create a client connection via Serial Port.

\end{itemize}

\sphinxAtStartPar
\sphinxstylestrong{conn\_mode}: (unused) Mode of a connection type.

\sphinxAtStartPar
\sphinxstylestrong{conn\_conf}: Configurations for making a connection. Depend on \sphinxstylestrong{conn\_type} (Type of Connection), there is a suitable configuration in JSON format as below.
\begin{quote}
\begin{itemize}
\item {} 
\sphinxAtStartPar
\sphinxstylestrong{TCPIPClient}

\end{itemize}

\begin{sphinxVerbatim}[commandchars=\\\{\}]
\PYG{p}{\PYGZob{}}
    \PYG{l+s+s2}{\PYGZdq{}}\PYG{l+s+s2}{address}\PYG{l+s+s2}{\PYGZdq{}}\PYG{p}{:} \PYG{p}{[}\PYG{n}{server} \PYG{n}{host}\PYG{p}{]}\PYG{p}{,} \PYG{c+c1}{\PYGZsh{} Optional. Default value is \PYGZdq{}localhost\PYGZdq{}.}
    \PYG{l+s+s2}{\PYGZdq{}}\PYG{l+s+s2}{port}\PYG{l+s+s2}{\PYGZdq{}}\PYG{p}{:} \PYG{p}{[}\PYG{n}{server} \PYG{n}{port}\PYG{p}{]}     \PYG{c+c1}{\PYGZsh{} Optional. Default value is 1234.}
    \PYG{l+s+s2}{\PYGZdq{}}\PYG{l+s+s2}{logfile}\PYG{l+s+s2}{\PYGZdq{}}\PYG{p}{:} \PYG{p}{[}\PYG{n}{Log} \PYG{n}{file} \PYG{n}{path}\PYG{o}{.} \PYG{n}{Possible} \PYG{n}{values}\PYG{p}{:} \PYG{l+s+s1}{\PYGZsq{}}\PYG{l+s+s1}{nonlog}\PYG{l+s+s1}{\PYGZsq{}}\PYG{p}{,} \PYG{l+s+s1}{\PYGZsq{}}\PYG{l+s+s1}{console}\PYG{l+s+s1}{\PYGZsq{}}\PYG{p}{,} \PYG{o}{\PYGZlt{}}\PYG{n}{user} \PYG{n}{define} \PYG{n}{path}\PYG{o}{\PYGZgt{}}\PYG{p}{]}
 \PYG{p}{\PYGZcb{}}
\end{sphinxVerbatim}
\begin{itemize}
\item {} 
\sphinxAtStartPar
\sphinxstylestrong{SSHClient}

\end{itemize}

\begin{sphinxVerbatim}[commandchars=\\\{\}]
\PYG{p}{\PYGZob{}}
    \PYG{l+s+s2}{\PYGZdq{}}\PYG{l+s+s2}{address}\PYG{l+s+s2}{\PYGZdq{}} \PYG{p}{:} \PYG{p}{[}\PYG{n}{server} \PYG{n}{host}\PYG{p}{]}\PYG{p}{,}  \PYG{c+c1}{\PYGZsh{} Optional. Default value is \PYGZdq{}localhost\PYGZdq{}.}
    \PYG{l+s+s2}{\PYGZdq{}}\PYG{l+s+s2}{port}\PYG{l+s+s2}{\PYGZdq{}} \PYG{p}{:} \PYG{p}{[}\PYG{n}{server} \PYG{n}{host}\PYG{p}{]}\PYG{p}{,}     \PYG{c+c1}{\PYGZsh{} Optional. Default value is 22.}
    \PYG{l+s+s2}{\PYGZdq{}}\PYG{l+s+s2}{username}\PYG{l+s+s2}{\PYGZdq{}} \PYG{p}{:} \PYG{p}{[}\PYG{n}{username}\PYG{p}{]}\PYG{p}{,}    \PYG{c+c1}{\PYGZsh{} Optional. Default value is \PYGZdq{}root\PYGZdq{}.}
    \PYG{l+s+s2}{\PYGZdq{}}\PYG{l+s+s2}{password}\PYG{l+s+s2}{\PYGZdq{}} \PYG{p}{:} \PYG{p}{[}\PYG{n}{password}\PYG{p}{]}\PYG{p}{,}    \PYG{c+c1}{\PYGZsh{} Optional. Default value is \PYGZdq{}\PYGZdq{}.}
    \PYG{l+s+s2}{\PYGZdq{}}\PYG{l+s+s2}{authentication}\PYG{l+s+s2}{\PYGZdq{}} \PYG{p}{:} \PYG{l+s+s2}{\PYGZdq{}}\PYG{l+s+s2}{password}\PYG{l+s+s2}{\PYGZdq{}} \PYG{o}{|} \PYG{l+s+s2}{\PYGZdq{}}\PYG{l+s+s2}{keyfile}\PYG{l+s+s2}{\PYGZdq{}} \PYG{o}{|} \PYG{l+s+s2}{\PYGZdq{}}\PYG{l+s+s2}{passwordkeyfile}\PYG{l+s+s2}{\PYGZdq{}}\PYG{p}{,}  \PYG{c+c1}{\PYGZsh{} Optional. Default value is \PYGZdq{}\PYGZdq{}.}
    \PYG{l+s+s2}{\PYGZdq{}}\PYG{l+s+s2}{key\PYGZus{}filename}\PYG{l+s+s2}{\PYGZdq{}} \PYG{p}{:} \PYG{p}{[}\PYG{n}{filename} \PYG{o+ow}{or} \PYG{n+nb}{list} \PYG{n}{of} \PYG{n}{filenames}\PYG{p}{]}\PYG{p}{,} \PYG{c+c1}{\PYGZsh{} Optional. Default value is null.}
    \PYG{l+s+s2}{\PYGZdq{}}\PYG{l+s+s2}{logfile}\PYG{l+s+s2}{\PYGZdq{}}\PYG{p}{:} \PYG{p}{[}\PYG{n}{Log} \PYG{n}{file} \PYG{n}{path}\PYG{o}{.} \PYG{n}{Possible} \PYG{n}{values}\PYG{p}{:} \PYG{l+s+s1}{\PYGZsq{}}\PYG{l+s+s1}{nonlog}\PYG{l+s+s1}{\PYGZsq{}}\PYG{p}{,} \PYG{l+s+s1}{\PYGZsq{}}\PYG{l+s+s1}{console}\PYG{l+s+s1}{\PYGZsq{}}\PYG{p}{,} \PYG{o}{\PYGZlt{}}\PYG{n}{user} \PYG{n}{define} \PYG{n}{path}\PYG{o}{\PYGZgt{}}\PYG{p}{]}
 \PYG{p}{\PYGZcb{}}
\end{sphinxVerbatim}
\begin{itemize}
\item {} 
\sphinxAtStartPar
\sphinxstylestrong{SerialClient}

\end{itemize}

\begin{sphinxVerbatim}[commandchars=\\\{\}]
\PYG{p}{\PYGZob{}}
    \PYG{l+s+s2}{\PYGZdq{}}\PYG{l+s+s2}{port}\PYG{l+s+s2}{\PYGZdq{}} \PYG{p}{:} \PYG{p}{[}\PYG{n}{comport} \PYG{o+ow}{or} \PYG{n}{null}\PYG{p}{]}\PYG{p}{,}
    \PYG{l+s+s2}{\PYGZdq{}}\PYG{l+s+s2}{baudrate}\PYG{l+s+s2}{\PYGZdq{}} \PYG{p}{:} \PYG{p}{[}\PYG{n}{Baud} \PYG{n}{rate} \PYG{n}{such} \PYG{k}{as} \PYG{l+m+mi}{9600} \PYG{o+ow}{or} \PYG{l+m+mi}{115200} \PYG{n}{etc}\PYG{o}{.}\PYG{p}{]}\PYG{p}{,}
    \PYG{l+s+s2}{\PYGZdq{}}\PYG{l+s+s2}{bytesize}\PYG{l+s+s2}{\PYGZdq{}} \PYG{p}{:} \PYG{p}{[}\PYG{n}{Number} \PYG{n}{of} \PYG{n}{data} \PYG{n}{bits}\PYG{o}{.} \PYG{n}{Possible} \PYG{n}{values}\PYG{p}{:} \PYG{l+m+mi}{5}\PYG{p}{,} \PYG{l+m+mi}{6}\PYG{p}{,} \PYG{l+m+mi}{7}\PYG{p}{,} \PYG{l+m+mi}{8}\PYG{p}{]}\PYG{p}{,}
    \PYG{l+s+s2}{\PYGZdq{}}\PYG{l+s+s2}{stopbits}\PYG{l+s+s2}{\PYGZdq{}} \PYG{p}{:} \PYG{p}{[}\PYG{n}{Number} \PYG{n}{of} \PYG{n}{stop} \PYG{n}{bits}\PYG{o}{.} \PYG{n}{Possible} \PYG{n}{values}\PYG{p}{:} \PYG{l+m+mi}{1}\PYG{p}{,} \PYG{l+m+mf}{1.5}\PYG{p}{,} \PYG{l+m+mi}{2}\PYG{p}{]}\PYG{p}{,}
    \PYG{l+s+s2}{\PYGZdq{}}\PYG{l+s+s2}{parity}\PYG{l+s+s2}{\PYGZdq{}} \PYG{p}{:} \PYG{p}{[}\PYG{n}{Enable} \PYG{n}{parity} \PYG{n}{checking}\PYG{o}{.} \PYG{n}{Possible} \PYG{n}{values}\PYG{p}{:} \PYG{l+s+s1}{\PYGZsq{}}\PYG{l+s+s1}{N}\PYG{l+s+s1}{\PYGZsq{}}\PYG{p}{,} \PYG{l+s+s1}{\PYGZsq{}}\PYG{l+s+s1}{E}\PYG{l+s+s1}{\PYGZsq{}}\PYG{p}{,} \PYG{l+s+s1}{\PYGZsq{}}\PYG{l+s+s1}{O}\PYG{l+s+s1}{\PYGZsq{}}\PYG{p}{,} \PYG{l+s+s1}{\PYGZsq{}}\PYG{l+s+s1}{M}\PYG{l+s+s1}{\PYGZsq{}}\PYG{p}{,} \PYG{l+s+s1}{\PYGZsq{}}\PYG{l+s+s1}{S}\PYG{l+s+s1}{\PYGZsq{}}\PYG{p}{]}\PYG{p}{,}
    \PYG{l+s+s2}{\PYGZdq{}}\PYG{l+s+s2}{rtscts}\PYG{l+s+s2}{\PYGZdq{}} \PYG{p}{:} \PYG{p}{[}\PYG{n}{Enable} \PYG{n}{hardware} \PYG{p}{(}\PYG{n}{RTS}\PYG{o}{/}\PYG{n}{CTS}\PYG{p}{)} \PYG{n}{flow} \PYG{n}{control}\PYG{o}{.}\PYG{p}{]}\PYG{p}{,}
    \PYG{l+s+s2}{\PYGZdq{}}\PYG{l+s+s2}{xonxoff}\PYG{l+s+s2}{\PYGZdq{}} \PYG{p}{:} \PYG{p}{[}\PYG{n}{Enable} \PYG{n}{software} \PYG{n}{flow} \PYG{n}{control}\PYG{o}{.}\PYG{p}{]}\PYG{p}{,}
    \PYG{l+s+s2}{\PYGZdq{}}\PYG{l+s+s2}{logfile}\PYG{l+s+s2}{\PYGZdq{}}\PYG{p}{:} \PYG{p}{[}\PYG{n}{Log} \PYG{n}{file} \PYG{n}{path}\PYG{o}{.} \PYG{n}{Possible} \PYG{n}{values}\PYG{p}{:} \PYG{l+s+s1}{\PYGZsq{}}\PYG{l+s+s1}{nonlog}\PYG{l+s+s1}{\PYGZsq{}}\PYG{p}{,} \PYG{l+s+s1}{\PYGZsq{}}\PYG{l+s+s1}{console}\PYG{l+s+s1}{\PYGZsq{}}\PYG{p}{,} \PYG{o}{\PYGZlt{}}\PYG{n}{user} \PYG{n}{define} \PYG{n}{path}\PYG{o}{\PYGZgt{}}\PYG{p}{]}
 \PYG{p}{\PYGZcb{}}
\end{sphinxVerbatim}
\end{quote}
\end{quote}
\end{quote}


\section{\sphinxstylestrong{disconnect}}
\label{\detokenize{index:disconnect}}\begin{quote}

\sphinxAtStartPar
\sphinxstylestrong{Use for disconnect a connection by name.}

\sphinxAtStartPar
\sphinxstylestrong{Syntax}:
\begin{quote}

\sphinxAtStartPar
\sphinxstylestrong{disconnect} \sphinxcode{\sphinxupquote{conn\_name}}
\end{quote}

\sphinxAtStartPar
\sphinxstylestrong{Arguments}:
\begin{quote}

\sphinxAtStartPar
\sphinxstylestrong{conn\_name}: Name of the connection.
\end{quote}
\end{quote}


\section{\sphinxstylestrong{send command}}
\label{\detokenize{index:send-command}}\begin{quote}

\sphinxAtStartPar
\sphinxstylestrong{Use for sending a command to the other side of connection.}

\sphinxAtStartPar
\sphinxstylestrong{Syntax}:
\begin{quote}

\sphinxAtStartPar
\sphinxstylestrong{send command} \sphinxcode{\sphinxupquote{{[}conn\_name{]}   {[}command{]}}} \sphinxstyleemphasis{(All parameters are
required to be in order)}

\sphinxAtStartPar
or

\sphinxAtStartPar
\sphinxstylestrong{send command}
\sphinxcode{\sphinxupquote{conn\_name={[}conn\_name{]}   command={[}command{]}}} \sphinxstyleemphasis{(All parameters are
assigned by name)} \#\#\#\#\# \sphinxstyleemphasis{Arguments}:
\end{quote}
\end{quote}
\begin{itemize}
\item {} 
\sphinxAtStartPar
\sphinxstylestrong{conn\_name}: Name of the connection.

\item {} 
\sphinxAtStartPar
\sphinxstylestrong{command}: Command to be sent.

\end{itemize}


\section{\sphinxstylestrong{verify}}
\label{\detokenize{index:verify}}\begin{quote}

\sphinxAtStartPar
\sphinxstylestrong{Use for verifying a response from the connection if it matched a pattern.}

\sphinxAtStartPar
\sphinxstylestrong{Syntax}:
\begin{quote}

\sphinxAtStartPar
\sphinxstylestrong{verify}
\sphinxcode{\sphinxupquote{{[}conn\_name{]}   {[}search\_pattern{]}   {[}timeout{]}   {[}fetch\_block{]}  {[}eob\_pattern{]} {[}filter\_pattern{]}  {[}send\_cmd{]}}}\sphinxstyleemphasis{(All
parameters are required to be in order)}

\sphinxAtStartPar
or

\sphinxAtStartPar
\sphinxstylestrong{verify}  \sphinxcode{\sphinxupquote{conn\_name={[}conn\_name{]}   search\_pattern={[}search\_pattern{]}  timeout={[}timeout{]}  fetch\_block={[}fetch\_block{]}  eob\_pattern={[}eob\_pattern{]} filter\_pattern={[}filter\_pattern{]}  send\_cmd={[}send\_cmd{]}}}
\sphinxstyleemphasis{(All parameters are assigned by name)}
\end{quote}

\sphinxAtStartPar
\sphinxstylestrong{Arguments}:
\begin{quote}

\sphinxAtStartPar
\sphinxstylestrong{conn\_name}: Name of the connection.

\sphinxAtStartPar
\sphinxstylestrong{search\_pattern}: Regular expression for matching with the response.

\sphinxAtStartPar
\sphinxstylestrong{timeout}: Timeout for waiting response matching pattern.

\sphinxAtStartPar
\sphinxstylestrong{fetch\_block}: If this value is true, every response line will be put into a block untill a line match \sphinxstylestrong{eob\_pattern} pattern.

\sphinxAtStartPar
\sphinxstylestrong{eob\_pattern}: Regular expression for matching the endline when using \sphinxstylestrong{fetch\_block}.

\sphinxAtStartPar
\sphinxstylestrong{filter\_pattern}: Regular expression for filtering every line of block when using \sphinxstylestrong{fetch\_block}.

\sphinxAtStartPar
\sphinxstylestrong{send\_cmd}: Command to be sent to the other side of connection and waiting for response.
\end{quote}

\sphinxAtStartPar
\sphinxstylestrong{Return value}:
\begin{quote}

\sphinxAtStartPar
\sphinxstylestrong{A corresponding match object if it is found.}

\sphinxAtStartPar
\sphinxstylestrong{E.g.}

\begin{sphinxVerbatim}[commandchars=\\\{\}]
\PYGZdl{}\PYGZob{}result\PYGZcb{} = verify  conn\PYGZus{}name=SSH\PYGZus{}Connection
                     search\PYGZus{}pattern=(?\PYGZlt{}=\PYGZbs{}s).*([0\PYGZhy{}9]..).*(command).\PYGZdl{}
                     send\PYGZus{}cmd=*echo This is the 1st test command.*
\end{sphinxVerbatim}
\begin{itemize}
\item {} 
\sphinxAtStartPar
\$\{result\}{[}0{]} will be \sphinxstylestrong{“This is the 1st test command.”} which is the matched string.

\item {} 
\sphinxAtStartPar
\$\{result\}{[}1{]} will be \sphinxstylestrong{“1st”} which is the first captured string.

\item {} 
\sphinxAtStartPar
\$\{result\}{[}2{]} will be \sphinxstylestrong{“command”} which is the second captured string.

\end{itemize}
\end{quote}
\end{quote}


\chapter{Example}
\label{\detokenize{index:example}}
\begin{sphinxVerbatim}[commandchars=\\\{\}]
*** Settings ***
Documentation    Suite description
Library     QConnectionLibrary.ConnectionManager

*** Test Cases ***
Test SSH Connection
    \PYGZsh{} Create config for connection.
    \PYGZdl{}\PYGZob{}config\PYGZus{}string\PYGZcb{}=    catenate
    ...  \PYGZob{}
    ...   \PYGZdq{}address\PYGZdq{}: \PYGZdq{}127.0.0.1\PYGZdq{},
    ...   \PYGZdq{}port\PYGZdq{}: 8022,
    ...   \PYGZdq{}username\PYGZdq{}: \PYGZdq{}root\PYGZdq{},
    ...   \PYGZdq{}password\PYGZdq{}: \PYGZdq{}\PYGZdq{},
    ...   \PYGZdq{}authentication\PYGZdq{}: \PYGZdq{}password\PYGZdq{},
    ...   \PYGZdq{}key\PYGZus{}filename\PYGZdq{}: null
    ...  \PYGZcb{}
    log to console       \PYGZbs{}nConnecting with configurations:\PYGZbs{}n\PYGZdl{}\PYGZob{}config\PYGZus{}string\PYGZcb{}
    \PYGZdl{}\PYGZob{}config\PYGZcb{}=             evaluate        json.loads(\PYGZsq{}\PYGZsq{}\PYGZsq{}\PYGZdl{}\PYGZob{}config\PYGZus{}string\PYGZcb{}\PYGZsq{}\PYGZsq{}\PYGZsq{})    json

    \PYGZsh{} Connect to the target with above configurations.
    connect             conn\PYGZus{}name=test\PYGZus{}ssh
    ...                 conn\PYGZus{}type=SSHClient
    ...                 conn\PYGZus{}conf=\PYGZdl{}\PYGZob{}config\PYGZcb{}

    \PYGZsh{} Send command \PYGZsq{}cd ..\PYGZsq{} and \PYGZsq{}ls\PYGZsq{} then wait for the response \PYGZsq{}.*\PYGZsq{} pattern.
    send command                conn\PYGZus{}name=test\PYGZus{}ssh
    ...                         command=cd ..

    \PYGZdl{}\PYGZob{}res\PYGZcb{}=     verify                  conn\PYGZus{}name=test\PYGZus{}ssh
    ...                                 search\PYGZus{}pattern=.*
    ...                                 send\PYGZus{}cmd=ls
    log to console     \PYGZdl{}\PYGZob{}res\PYGZcb{}

    \PYGZsh{} Disconnect
    disconnect  test\PYGZus{}ssh
\end{sphinxVerbatim}


\chapter{Contribution Guidelines}
\label{\detokenize{index:contribution-guidelines}}
\sphinxAtStartPar
QConnectBaseLibrary is designed for ease of making an extension library. By that way you can take advantage of the QConnectBaseLibrary’s
infrastructure for handling your own connection protocal. For creating an extension library for QConnectBaseLibrary, please following below
steps.
\begin{enumerate}
\sphinxsetlistlabels{\arabic}{enumi}{enumii}{}{.}%
\item {} 
\sphinxAtStartPar
Create a library package which have the prefix name is \sphinxstylestrong{robotframework\sphinxhyphen{}qconnect\sphinxhyphen{}}\sphinxstyleemphasis{{[}your specific name{]}}.

\item {} 
\sphinxAtStartPar
Your hadling connection class should be derived from \sphinxstylestrong{QConnectionLibrary.connection\_base.ConnectionBase}  class.

\item {} 
\sphinxAtStartPar
In your \sphinxstyleemphasis{Connection Class}, override below attributes and methods:

\end{enumerate}
\begin{quote}
\begin{itemize}
\item {} 
\sphinxAtStartPar
\sphinxstylestrong{\_CONNECTION\_TYPE}: name of your connection type. It will be the input of the conn\_type argument when using \sphinxstylestrong{connect} keyword. Depend on the type name, the library will detemine the correct connection handling class.

\item {} 
\sphinxAtStartPar
\sphinxstylestrong{\_\_init\_\_(self, \_mode, config)}: in this constructor method, you should:

\end{itemize}
\begin{itemize}
\item {} 
\sphinxAtStartPar
Prepare resource for your connection.

\item {} 
\sphinxAtStartPar
Initialize receiver thread by calling \sphinxstylestrong{self.\_init\_thread\_receiver(cls.\_socket\_instance, mode=””)} method.

\item {} 
\sphinxAtStartPar
Configure and initialize the lowlevel receiver thread (if it’s necessary) as below

\begin{sphinxVerbatim}[commandchars=\\\{\}]
\PYG{n+nb+bp}{self}\PYG{o}{.}\PYG{n}{\PYGZus{}llrecv\PYGZus{}thrd\PYGZus{}obj} \PYG{o}{=} \PYG{k+kc}{None}
 \PYG{n+nb+bp}{self}\PYG{o}{.}\PYG{n}{\PYGZus{}llrecv\PYGZus{}thrd\PYGZus{}term} \PYG{o}{=} \PYG{n}{threading}\PYG{o}{.}\PYG{n}{Event}\PYG{p}{(}\PYG{p}{)}
 \PYG{n+nb+bp}{self}\PYG{o}{.}\PYG{n}{\PYGZus{}init\PYGZus{}thrd\PYGZus{}llrecv}\PYG{p}{(}\PYG{n+nb+bp}{cls}\PYG{o}{.}\PYG{n}{\PYGZus{}socket\PYGZus{}instance}\PYG{p}{)}
\end{sphinxVerbatim}

\item {} 
\sphinxAtStartPar
Incase you use the lowlevel receiver thread. You should implement the \sphinxstylestrong{thrd\_llrecv\_from\_connection\_interface()} method. This method is a mediate layer which will receive the data from connection at the very beginning, do some process then put them in a queue for the \sphinxstylestrong{receiver thread} above getting later.

\item {} 
\sphinxAtStartPar
Create the queue for this connection (use Queue.Queue).

\end{itemize}
\begin{itemize}
\item {} 
\sphinxAtStartPar
\sphinxstylestrong{connect()}: implement the way you use to make your own connection protocol.

\item {} 
\sphinxAtStartPar
\sphinxstylestrong{\_read()}: implement the way to receive data from connection.

\item {} 
\sphinxAtStartPar
\sphinxstylestrong{\_write()}: implement the way to send data via connection.

\item {} 
\sphinxAtStartPar
\sphinxstylestrong{disconnect()}: implement the way you use to disconnect your own connection protocol.

\item {} 
\sphinxAtStartPar
\sphinxstylestrong{quit()}: implement the way you use to quit connection and clean resource.

\end{itemize}
\end{quote}


\chapter{Configure Git and correct EOL handling}
\label{\detokenize{index:configure-git-and-correct-eol-handling}}
\sphinxAtStartPar
Here you can find the references for \sphinxhref{https://help.github.com/articles/dealing-with-line-endings/}{Dealing with line
endings}.

\sphinxAtStartPar
Every time you press return on your keyboard you’re actually inserting
an invisible character called a line ending. Historically, different
operating systems have handled line endings differently. When you view
changes in a file, Git handles line endings in its own way. Since you’re
collaborating on projects with Git and GitHub, Git might produce
unexpected results if, for example, you’re working on a Windows machine,
and your collaborator has made a change in OS X.

\sphinxAtStartPar
To avoid problems in your diffs, you can configure Git to properly
handle line endings. If you are storing the .gitattributes file directly
inside of your repository, than you can asure that all EOL are manged by
git correctly as defined.


\chapter{Sourcecode Documentation}
\label{\detokenize{index:sourcecode-documentation}}
\sphinxAtStartPar
For investigating sourcecode, please refer to \sphinxhref{docs/html/index.html}{QConnectBaseLibrary
Documentation}


\chapter{Feedback}
\label{\detokenize{index:feedback}}
\sphinxAtStartPar
If you have any problem when using the library or think there is a
better solution for any part of the library, I’d love to know it, as
this will all help me to improve the library. Connect with me at
\sphinxhref{mailto:cuong.nguyenhuynhtri@vn.bosch.com}{cuong.nguyenhuynhtri@vn.bosch.com}.

\sphinxAtStartPar
Do share your valuable opinion, I appreciate your honest feedback!


\chapter{About}
\label{\detokenize{index:about}}

\section{Maintainers}
\label{\detokenize{index:maintainers}}
\sphinxAtStartPar
\sphinxhref{mailto:cuong.nguyenhuynhtri@vn.bosch.com}{Nguyen Huynh Tri Cuong}


\section{Contributors}
\label{\detokenize{index:contributors}}
\sphinxAtStartPar
\sphinxhref{mailto:cuong.nguyenhuynhtri@vn.bosch.com}{Nguyen Huynh Tri Cuong}

\sphinxAtStartPar
\sphinxhref{mailto:thomas.pollerspoeck@de.bosch.com}{Thomas Pollerspoeck}


\section{3rd Party Licenses}
\label{\detokenize{index:rd-party-licenses}}
\sphinxAtStartPar
You must mention all 3rd party licenses (e.g. OSS) licenses used by your
project here. Example:


\begin{savenotes}\sphinxattablestart
\centering
\begin{tabulary}{\linewidth}[t]{|T|T|T|}
\hline
\sphinxstyletheadfamily 
\sphinxAtStartPar
Name
&\sphinxstyletheadfamily 
\sphinxAtStartPar
License
&\sphinxstyletheadfamily 
\sphinxAtStartPar
Type
\\
\hline
\sphinxAtStartPar
\sphinxhref{http://felix.apache.org/}{Apache Felix}.
&
\sphinxAtStartPar
\sphinxhref{http://www.apache.org/licenses/LICENSE-2.0.txt}{Apache 2.0 License}.
&
\sphinxAtStartPar
Dependency
\\
\hline
\end{tabulary}
\par
\sphinxattableend\end{savenotes}


\section{Used Encryption}
\label{\detokenize{index:used-encryption}}
\sphinxAtStartPar
Declaration of the usage of any encryption (see BIOS Repository Policy
\S{}4.a).


\section{License}
\label{\detokenize{index:license}}
\sphinxAtStartPar
{\hyperref[\detokenize{index:license}]{\emph{\sphinxincludegraphics{{D:\Project\robot\ROBFW-229\robotframework-qconnect-base\doc\_build\doctrees\images\69d3e9dd766a3c0be6304e090529292e75d3a979\bioslv4-badge}.svg}}}}
\begin{quote}

\sphinxAtStartPar
Copyright (c) 2009, 2018 Robert Bosch GmbH and its subsidiaries. This
program and the accompanying materials are made available under the
terms of the Bosch Internal Open Source License v4 which accompanies
this distribution, and is available at
\sphinxurl{http://bios.intranet.bosch.com/bioslv4.txt}
\end{quote}




\chapter{QConnect base library’s API!}
\label{\detokenize{index:qconnect-base-library-s-api}}

\section{QConnectionLibrary package}
\label{\detokenize{QConnectionLibrary:qconnectionlibrary-package}}\label{\detokenize{QConnectionLibrary::doc}}

\subsection{Module contents}
\label{\detokenize{QConnectionLibrary:module-QConnectionLibrary.connection_manager}}\label{\detokenize{QConnectionLibrary:module-contents}}\index{module@\spxentry{module}!QConnectionLibrary.connection\_manager@\spxentry{QConnectionLibrary.connection\_manager}}\index{QConnectionLibrary.connection\_manager@\spxentry{QConnectionLibrary.connection\_manager}!module@\spxentry{module}}\index{ConnectParam (class in QConnectionLibrary.connection\_manager)@\spxentry{ConnectParam}\spxextra{class in QConnectionLibrary.connection\_manager}}

\begin{fulllineitems}
\phantomsection\label{\detokenize{QConnectionLibrary:QConnectionLibrary.connection_manager.ConnectParam}}\pysiglinewithargsret{\sphinxbfcode{\sphinxupquote{class }}\sphinxcode{\sphinxupquote{QConnectionLibrary.connection\_manager.}}\sphinxbfcode{\sphinxupquote{ConnectParam}}}{\emph{\DUrole{o}{**}\DUrole{n}{dictionary}}}{}
\sphinxAtStartPar
Bases: {\hyperref[\detokenize{QConnectionLibrary:QConnectionLibrary.connection_manager.InputParam}]{\sphinxcrossref{\sphinxcode{\sphinxupquote{QConnectionLibrary.connection\_manager.InputParam}}}}}

\sphinxAtStartPar
Class for storing parameters for connect action.
\index{conn\_conf (QConnectionLibrary.connection\_manager.ConnectParam attribute)@\spxentry{conn\_conf}\spxextra{QConnectionLibrary.connection\_manager.ConnectParam attribute}}

\begin{fulllineitems}
\phantomsection\label{\detokenize{QConnectionLibrary:QConnectionLibrary.connection_manager.ConnectParam.conn_conf}}\pysigline{\sphinxbfcode{\sphinxupquote{conn\_conf}}\sphinxbfcode{\sphinxupquote{ = \{\}}}}
\end{fulllineitems}

\index{conn\_mode (QConnectionLibrary.connection\_manager.ConnectParam attribute)@\spxentry{conn\_mode}\spxextra{QConnectionLibrary.connection\_manager.ConnectParam attribute}}

\begin{fulllineitems}
\phantomsection\label{\detokenize{QConnectionLibrary:QConnectionLibrary.connection_manager.ConnectParam.conn_mode}}\pysigline{\sphinxbfcode{\sphinxupquote{conn\_mode}}\sphinxbfcode{\sphinxupquote{ = \textquotesingle{}\textquotesingle{}}}}
\end{fulllineitems}

\index{conn\_name (QConnectionLibrary.connection\_manager.ConnectParam attribute)@\spxentry{conn\_name}\spxextra{QConnectionLibrary.connection\_manager.ConnectParam attribute}}

\begin{fulllineitems}
\phantomsection\label{\detokenize{QConnectionLibrary:QConnectionLibrary.connection_manager.ConnectParam.conn_name}}\pysigline{\sphinxbfcode{\sphinxupquote{conn\_name}}\sphinxbfcode{\sphinxupquote{ = \textquotesingle{}default\_conn\textquotesingle{}}}}
\end{fulllineitems}

\index{conn\_type (QConnectionLibrary.connection\_manager.ConnectParam attribute)@\spxentry{conn\_type}\spxextra{QConnectionLibrary.connection\_manager.ConnectParam attribute}}

\begin{fulllineitems}
\phantomsection\label{\detokenize{QConnectionLibrary:QConnectionLibrary.connection_manager.ConnectParam.conn_type}}\pysigline{\sphinxbfcode{\sphinxupquote{conn\_type}}\sphinxbfcode{\sphinxupquote{ = \textquotesingle{}TCPIP\textquotesingle{}}}}
\end{fulllineitems}

\index{exclude\_list (QConnectionLibrary.connection\_manager.ConnectParam attribute)@\spxentry{exclude\_list}\spxextra{QConnectionLibrary.connection\_manager.ConnectParam attribute}}

\begin{fulllineitems}
\phantomsection\label{\detokenize{QConnectionLibrary:QConnectionLibrary.connection_manager.ConnectParam.exclude_list}}\pysigline{\sphinxbfcode{\sphinxupquote{exclude\_list}}\sphinxbfcode{\sphinxupquote{ = {[}\textquotesingle{}conn\_conf\textquotesingle{}{]}}}}
\end{fulllineitems}

\index{id (QConnectionLibrary.connection\_manager.ConnectParam attribute)@\spxentry{id}\spxextra{QConnectionLibrary.connection\_manager.ConnectParam attribute}}

\begin{fulllineitems}
\phantomsection\label{\detokenize{QConnectionLibrary:QConnectionLibrary.connection_manager.ConnectParam.id}}\pysigline{\sphinxbfcode{\sphinxupquote{id}}\sphinxbfcode{\sphinxupquote{ = 0}}}
\end{fulllineitems}


\end{fulllineitems}

\index{ConnectionManager (class in QConnectionLibrary.connection\_manager)@\spxentry{ConnectionManager}\spxextra{class in QConnectionLibrary.connection\_manager}}

\begin{fulllineitems}
\phantomsection\label{\detokenize{QConnectionLibrary:QConnectionLibrary.connection_manager.ConnectionManager}}\pysiglinewithargsret{\sphinxbfcode{\sphinxupquote{class }}\sphinxcode{\sphinxupquote{QConnectionLibrary.connection\_manager.}}\sphinxbfcode{\sphinxupquote{ConnectionManager}}}{\emph{\DUrole{o}{*}\DUrole{n}{args}}, \emph{\DUrole{o}{**}\DUrole{n}{kwargs}}}{}
\sphinxAtStartPar
Bases: \sphinxcode{\sphinxupquote{QConnectionLibrary.utils.Singleton}}

\sphinxAtStartPar
Class to manage all connections.
\index{LIBRARY\_EXTENSION\_PREFIX (QConnectionLibrary.connection\_manager.ConnectionManager attribute)@\spxentry{LIBRARY\_EXTENSION\_PREFIX}\spxextra{QConnectionLibrary.connection\_manager.ConnectionManager attribute}}

\begin{fulllineitems}
\phantomsection\label{\detokenize{QConnectionLibrary:QConnectionLibrary.connection_manager.ConnectionManager.LIBRARY_EXTENSION_PREFIX}}\pysigline{\sphinxbfcode{\sphinxupquote{LIBRARY\_EXTENSION\_PREFIX}}\sphinxbfcode{\sphinxupquote{ = \textquotesingle{}robotframework\_qconnect\textquotesingle{}}}}
\end{fulllineitems}

\index{LIBRARY\_EXTENSION\_PREFIX2 (QConnectionLibrary.connection\_manager.ConnectionManager attribute)@\spxentry{LIBRARY\_EXTENSION\_PREFIX2}\spxextra{QConnectionLibrary.connection\_manager.ConnectionManager attribute}}

\begin{fulllineitems}
\phantomsection\label{\detokenize{QConnectionLibrary:QConnectionLibrary.connection_manager.ConnectionManager.LIBRARY_EXTENSION_PREFIX2}}\pysigline{\sphinxbfcode{\sphinxupquote{LIBRARY\_EXTENSION\_PREFIX2}}\sphinxbfcode{\sphinxupquote{ = \textquotesingle{}QConnection\textquotesingle{}}}}
\end{fulllineitems}

\index{ROBOT\_AUTO\_KEYWORDS (QConnectionLibrary.connection\_manager.ConnectionManager attribute)@\spxentry{ROBOT\_AUTO\_KEYWORDS}\spxextra{QConnectionLibrary.connection\_manager.ConnectionManager attribute}}

\begin{fulllineitems}
\phantomsection\label{\detokenize{QConnectionLibrary:QConnectionLibrary.connection_manager.ConnectionManager.ROBOT_AUTO_KEYWORDS}}\pysigline{\sphinxbfcode{\sphinxupquote{ROBOT\_AUTO\_KEYWORDS}}\sphinxbfcode{\sphinxupquote{ = False}}}
\end{fulllineitems}

\index{ROBOT\_LIBRARY\_SCOPE (QConnectionLibrary.connection\_manager.ConnectionManager attribute)@\spxentry{ROBOT\_LIBRARY\_SCOPE}\spxextra{QConnectionLibrary.connection\_manager.ConnectionManager attribute}}

\begin{fulllineitems}
\phantomsection\label{\detokenize{QConnectionLibrary:QConnectionLibrary.connection_manager.ConnectionManager.ROBOT_LIBRARY_SCOPE}}\pysigline{\sphinxbfcode{\sphinxupquote{ROBOT\_LIBRARY\_SCOPE}}\sphinxbfcode{\sphinxupquote{ = \textquotesingle{}GLOBAL\textquotesingle{}}}}
\end{fulllineitems}

\index{add\_connection() (QConnectionLibrary.connection\_manager.ConnectionManager method)@\spxentry{add\_connection()}\spxextra{QConnectionLibrary.connection\_manager.ConnectionManager method}}

\begin{fulllineitems}
\phantomsection\label{\detokenize{QConnectionLibrary:QConnectionLibrary.connection_manager.ConnectionManager.add_connection}}\pysiglinewithargsret{\sphinxbfcode{\sphinxupquote{add\_connection}}}{\emph{\DUrole{n}{name}}, \emph{\DUrole{n}{conn}}}{}
\sphinxAtStartPar
Add a connection to managed dictionary.
\begin{description}
\item[{Args:}] \leavevmode
\sphinxAtStartPar
name: connection’s name.

\sphinxAtStartPar
conn: connection object.

\item[{Returns:}] \leavevmode
\sphinxAtStartPar
None.

\end{description}

\end{fulllineitems}

\index{connect() (QConnectionLibrary.connection\_manager.ConnectionManager method)@\spxentry{connect()}\spxextra{QConnectionLibrary.connection\_manager.ConnectionManager method}}

\begin{fulllineitems}
\phantomsection\label{\detokenize{QConnectionLibrary:QConnectionLibrary.connection_manager.ConnectionManager.connect}}\pysiglinewithargsret{\sphinxbfcode{\sphinxupquote{connect}}}{\emph{\DUrole{o}{*}\DUrole{n}{args}}, \emph{\DUrole{o}{**}\DUrole{n}{kwargs}}}{}
\sphinxAtStartPar
Keyword for making a connection.
\begin{description}
\item[{Args:}] \leavevmode
\sphinxAtStartPar
args:   Non\sphinxhyphen{}Keyword Arguments.

\sphinxAtStartPar
kwargs:   Keyword Arguments.

\item[{Returns:}] \leavevmode
\sphinxAtStartPar
None.

\end{description}

\end{fulllineitems}

\index{connect\_named\_args() (QConnectionLibrary.connection\_manager.ConnectionManager method)@\spxentry{connect\_named\_args()}\spxextra{QConnectionLibrary.connection\_manager.ConnectionManager method}}

\begin{fulllineitems}
\phantomsection\label{\detokenize{QConnectionLibrary:QConnectionLibrary.connection_manager.ConnectionManager.connect_named_args}}\pysiglinewithargsret{\sphinxbfcode{\sphinxupquote{connect\_named\_args}}}{\emph{\DUrole{o}{**}\DUrole{n}{kwargs}}}{}
\sphinxAtStartPar
Making a connection with name arguments.
\begin{description}
\item[{Args:}] \leavevmode
\sphinxAtStartPar
kwargs: Dictionary of arguments.

\item[{Returns:}] \leavevmode
\sphinxAtStartPar
None.

\end{description}

\end{fulllineitems}

\index{connect\_unnamed\_args() (QConnectionLibrary.connection\_manager.ConnectionManager method)@\spxentry{connect\_unnamed\_args()}\spxextra{QConnectionLibrary.connection\_manager.ConnectionManager method}}

\begin{fulllineitems}
\phantomsection\label{\detokenize{QConnectionLibrary:QConnectionLibrary.connection_manager.ConnectionManager.connect_unnamed_args}}\pysiglinewithargsret{\sphinxbfcode{\sphinxupquote{connect\_unnamed\_args}}}{\emph{\DUrole{n}{connection\_name}}, \emph{\DUrole{n}{connection\_type}}, \emph{\DUrole{n}{mode}}, \emph{\DUrole{n}{config}}}{}
\sphinxAtStartPar
Making a connection.
\begin{description}
\item[{Args:}] \leavevmode
\sphinxAtStartPar
connection\_name: Name of connection.

\sphinxAtStartPar
connection\_type: Type of connection.

\sphinxAtStartPar
mode: Connection mode.

\sphinxAtStartPar
config: Configuration for connection.

\item[{Returns:}] \leavevmode
\sphinxAtStartPar
None.

\end{description}

\end{fulllineitems}

\index{disconnect() (QConnectionLibrary.connection\_manager.ConnectionManager method)@\spxentry{disconnect()}\spxextra{QConnectionLibrary.connection\_manager.ConnectionManager method}}

\begin{fulllineitems}
\phantomsection\label{\detokenize{QConnectionLibrary:QConnectionLibrary.connection_manager.ConnectionManager.disconnect}}\pysiglinewithargsret{\sphinxbfcode{\sphinxupquote{disconnect}}}{\emph{\DUrole{n}{connection\_name}}}{}
\sphinxAtStartPar
Keyword for disconnecting a connection by name.
\begin{description}
\item[{Args:}] \leavevmode
\sphinxAtStartPar
connection\_name: Name of connection.

\item[{Returns:}] \leavevmode
\sphinxAtStartPar
None.

\end{description}

\end{fulllineitems}

\index{get\_connection\_by\_name() (QConnectionLibrary.connection\_manager.ConnectionManager method)@\spxentry{get\_connection\_by\_name()}\spxextra{QConnectionLibrary.connection\_manager.ConnectionManager method}}

\begin{fulllineitems}
\phantomsection\label{\detokenize{QConnectionLibrary:QConnectionLibrary.connection_manager.ConnectionManager.get_connection_by_name}}\pysiglinewithargsret{\sphinxbfcode{\sphinxupquote{get\_connection\_by\_name}}}{\emph{\DUrole{n}{connection\_name}}}{}
\sphinxAtStartPar
Get an exist connection by name.
\begin{description}
\item[{Args:}] \leavevmode
\sphinxAtStartPar
connection\_name: connection’s name.

\item[{Returns:}] \leavevmode
\sphinxAtStartPar
Connection object.

\end{description}

\end{fulllineitems}

\index{quit() (QConnectionLibrary.connection\_manager.ConnectionManager method)@\spxentry{quit()}\spxextra{QConnectionLibrary.connection\_manager.ConnectionManager method}}

\begin{fulllineitems}
\phantomsection\label{\detokenize{QConnectionLibrary:QConnectionLibrary.connection_manager.ConnectionManager.quit}}\pysiglinewithargsret{\sphinxbfcode{\sphinxupquote{quit}}}{}{}
\sphinxAtStartPar
Quit connection manager.
\begin{description}
\item[{Returns:}] \leavevmode
\sphinxAtStartPar
None.

\end{description}

\end{fulllineitems}

\index{remove\_connection() (QConnectionLibrary.connection\_manager.ConnectionManager method)@\spxentry{remove\_connection()}\spxextra{QConnectionLibrary.connection\_manager.ConnectionManager method}}

\begin{fulllineitems}
\phantomsection\label{\detokenize{QConnectionLibrary:QConnectionLibrary.connection_manager.ConnectionManager.remove_connection}}\pysiglinewithargsret{\sphinxbfcode{\sphinxupquote{remove\_connection}}}{\emph{\DUrole{n}{connection\_name}}}{}
\sphinxAtStartPar
Remove a connection by name.
\begin{description}
\item[{Args:}] \leavevmode
\sphinxAtStartPar
connection\_name: connection name.

\item[{Returns:}] \leavevmode
\sphinxAtStartPar
None.

\end{description}

\end{fulllineitems}

\index{send\_command() (QConnectionLibrary.connection\_manager.ConnectionManager method)@\spxentry{send\_command()}\spxextra{QConnectionLibrary.connection\_manager.ConnectionManager method}}

\begin{fulllineitems}
\phantomsection\label{\detokenize{QConnectionLibrary:QConnectionLibrary.connection_manager.ConnectionManager.send_command}}\pysiglinewithargsret{\sphinxbfcode{\sphinxupquote{send\_command}}}{\emph{\DUrole{o}{*}\DUrole{n}{args}}, \emph{\DUrole{o}{**}\DUrole{n}{kwargs}}}{}
\sphinxAtStartPar
Keyword for sending command to a connection.
\begin{description}
\item[{Args:}] \leavevmode
\sphinxAtStartPar
args:   Non\sphinxhyphen{}Keyword Arguments.

\sphinxAtStartPar
kwargs:   Keyword Arguments.

\item[{Returns:}] \leavevmode
\sphinxAtStartPar
None.

\end{description}

\end{fulllineitems}

\index{send\_command\_named\_args() (QConnectionLibrary.connection\_manager.ConnectionManager method)@\spxentry{send\_command\_named\_args()}\spxextra{QConnectionLibrary.connection\_manager.ConnectionManager method}}

\begin{fulllineitems}
\phantomsection\label{\detokenize{QConnectionLibrary:QConnectionLibrary.connection_manager.ConnectionManager.send_command_named_args}}\pysiglinewithargsret{\sphinxbfcode{\sphinxupquote{send\_command\_named\_args}}}{\emph{\DUrole{o}{**}\DUrole{n}{args}}}{}
\sphinxAtStartPar
Send command to a connection with name arguments.
\begin{description}
\item[{Args:}] \leavevmode
\sphinxAtStartPar
args: Dictionary of arguments.

\item[{Returns:}] \leavevmode
\sphinxAtStartPar
None.

\end{description}

\end{fulllineitems}

\index{send\_command\_unnamed\_args() (QConnectionLibrary.connection\_manager.ConnectionManager method)@\spxentry{send\_command\_unnamed\_args()}\spxextra{QConnectionLibrary.connection\_manager.ConnectionManager method}}

\begin{fulllineitems}
\phantomsection\label{\detokenize{QConnectionLibrary:QConnectionLibrary.connection_manager.ConnectionManager.send_command_unnamed_args}}\pysiglinewithargsret{\sphinxbfcode{\sphinxupquote{send\_command\_unnamed\_args}}}{\emph{\DUrole{n}{connection\_name}}, \emph{\DUrole{n}{command}}}{}
\sphinxAtStartPar
Send command to a connection.
\begin{description}
\item[{Args:}] \leavevmode
\sphinxAtStartPar
connection\_name: connection’s name.

\sphinxAtStartPar
command: command.

\item[{Returns:}] \leavevmode
\sphinxAtStartPar
None.

\end{description}

\end{fulllineitems}

\index{verify() (QConnectionLibrary.connection\_manager.ConnectionManager method)@\spxentry{verify()}\spxextra{QConnectionLibrary.connection\_manager.ConnectionManager method}}

\begin{fulllineitems}
\phantomsection\label{\detokenize{QConnectionLibrary:QConnectionLibrary.connection_manager.ConnectionManager.verify}}\pysiglinewithargsret{\sphinxbfcode{\sphinxupquote{verify}}}{\emph{\DUrole{o}{*}\DUrole{n}{args}}, \emph{\DUrole{o}{**}\DUrole{n}{kwargs}}}{}
\sphinxAtStartPar
Keyword uses to verify a pattern from connection response after sending a command.
\begin{description}
\item[{Args:}] \leavevmode
\sphinxAtStartPar
args:   Non\sphinxhyphen{}Keyword Arguments.

\sphinxAtStartPar
kwargs:   Keyword Arguments.

\item[{Returns:}] \leavevmode
\sphinxAtStartPar
match\_res: matched string.

\end{description}

\end{fulllineitems}

\index{verify\_named\_args() (QConnectionLibrary.connection\_manager.ConnectionManager method)@\spxentry{verify\_named\_args()}\spxextra{QConnectionLibrary.connection\_manager.ConnectionManager method}}

\begin{fulllineitems}
\phantomsection\label{\detokenize{QConnectionLibrary:QConnectionLibrary.connection_manager.ConnectionManager.verify_named_args}}\pysiglinewithargsret{\sphinxbfcode{\sphinxupquote{verify\_named\_args}}}{\emph{\DUrole{o}{**}\DUrole{n}{kwargs}}}{}
\sphinxAtStartPar
Verify a pattern from connection response after sending a command with named arguments.
\begin{description}
\item[{Args:}] \leavevmode
\sphinxAtStartPar
kwargs: Dictionary of arguments.

\item[{Returns:}] \leavevmode
\sphinxAtStartPar
match\_res: matched string.

\end{description}

\end{fulllineitems}

\index{verify\_unnamed\_args() (QConnectionLibrary.connection\_manager.ConnectionManager method)@\spxentry{verify\_unnamed\_args()}\spxextra{QConnectionLibrary.connection\_manager.ConnectionManager method}}

\begin{fulllineitems}
\phantomsection\label{\detokenize{QConnectionLibrary:QConnectionLibrary.connection_manager.ConnectionManager.verify_unnamed_args}}\pysiglinewithargsret{\sphinxbfcode{\sphinxupquote{verify\_unnamed\_args}}}{\emph{\DUrole{n}{connection\_name}}, \emph{\DUrole{n}{search\_obj}}, \emph{\DUrole{n}{timeout}\DUrole{o}{=}\DUrole{default_value}{0}}, \emph{\DUrole{n}{fetch\_block}\DUrole{o}{=}\DUrole{default_value}{False}}, \emph{\DUrole{n}{eob\_pattern}\DUrole{o}{=}\DUrole{default_value}{\textquotesingle{}.*\textquotesingle{}}}, \emph{\DUrole{n}{filter\_pattern}\DUrole{o}{=}\DUrole{default_value}{\textquotesingle{}.*\textquotesingle{}}}, \emph{\DUrole{o}{*}\DUrole{n}{fct\_args}}}{}
\sphinxAtStartPar
Verify a pattern from connection response after sending a command.
\begin{description}
\item[{Args:}] \leavevmode
\sphinxAtStartPar
connection\_name: connection’s name.

\sphinxAtStartPar
search\_obj: search pattern.

\sphinxAtStartPar
timeout: timeout for waiting result.

\sphinxAtStartPar
fetch\_block: use fetch block feature.

\sphinxAtStartPar
end\_of\_block\_pattern: pattern for detecting the end of block.

\sphinxAtStartPar
filter\_pattern: line filter pattern.

\sphinxAtStartPar
fct\_args: command to be sent.

\item[{Returns:}] \leavevmode
\sphinxAtStartPar
match\_res: matched string.

\end{description}

\end{fulllineitems}


\end{fulllineitems}

\index{InputParam (class in QConnectionLibrary.connection\_manager)@\spxentry{InputParam}\spxextra{class in QConnectionLibrary.connection\_manager}}

\begin{fulllineitems}
\phantomsection\label{\detokenize{QConnectionLibrary:QConnectionLibrary.connection_manager.InputParam}}\pysiglinewithargsret{\sphinxbfcode{\sphinxupquote{class }}\sphinxcode{\sphinxupquote{QConnectionLibrary.connection\_manager.}}\sphinxbfcode{\sphinxupquote{InputParam}}}{\emph{\DUrole{o}{**}\DUrole{n}{dictionary}}}{}
\sphinxAtStartPar
Bases: \sphinxcode{\sphinxupquote{QConnectionLibrary.utils.DictToClass}}
\index{get\_attr\_list() (QConnectionLibrary.connection\_manager.InputParam class method)@\spxentry{get\_attr\_list()}\spxextra{QConnectionLibrary.connection\_manager.InputParam class method}}

\begin{fulllineitems}
\phantomsection\label{\detokenize{QConnectionLibrary:QConnectionLibrary.connection_manager.InputParam.get_attr_list}}\pysiglinewithargsret{\sphinxbfcode{\sphinxupquote{classmethod }}\sphinxbfcode{\sphinxupquote{get\_attr\_list}}}{}{}
\end{fulllineitems}


\end{fulllineitems}

\index{SendCommandParam (class in QConnectionLibrary.connection\_manager)@\spxentry{SendCommandParam}\spxextra{class in QConnectionLibrary.connection\_manager}}

\begin{fulllineitems}
\phantomsection\label{\detokenize{QConnectionLibrary:QConnectionLibrary.connection_manager.SendCommandParam}}\pysiglinewithargsret{\sphinxbfcode{\sphinxupquote{class }}\sphinxcode{\sphinxupquote{QConnectionLibrary.connection\_manager.}}\sphinxbfcode{\sphinxupquote{SendCommandParam}}}{\emph{\DUrole{o}{**}\DUrole{n}{dictionary}}}{}
\sphinxAtStartPar
Bases: {\hyperref[\detokenize{QConnectionLibrary:QConnectionLibrary.connection_manager.InputParam}]{\sphinxcrossref{\sphinxcode{\sphinxupquote{QConnectionLibrary.connection\_manager.InputParam}}}}}

\sphinxAtStartPar
Class for storing parameters for send command action.
\index{command (QConnectionLibrary.connection\_manager.SendCommandParam attribute)@\spxentry{command}\spxextra{QConnectionLibrary.connection\_manager.SendCommandParam attribute}}

\begin{fulllineitems}
\phantomsection\label{\detokenize{QConnectionLibrary:QConnectionLibrary.connection_manager.SendCommandParam.command}}\pysigline{\sphinxbfcode{\sphinxupquote{command}}\sphinxbfcode{\sphinxupquote{ = \textquotesingle{}\textquotesingle{}}}}
\end{fulllineitems}

\index{conn\_name (QConnectionLibrary.connection\_manager.SendCommandParam attribute)@\spxentry{conn\_name}\spxextra{QConnectionLibrary.connection\_manager.SendCommandParam attribute}}

\begin{fulllineitems}
\phantomsection\label{\detokenize{QConnectionLibrary:QConnectionLibrary.connection_manager.SendCommandParam.conn_name}}\pysigline{\sphinxbfcode{\sphinxupquote{conn\_name}}\sphinxbfcode{\sphinxupquote{ = \textquotesingle{}default\_conn\textquotesingle{}}}}
\end{fulllineitems}


\end{fulllineitems}

\index{TestOption (class in QConnectionLibrary.connection\_manager)@\spxentry{TestOption}\spxextra{class in QConnectionLibrary.connection\_manager}}

\begin{fulllineitems}
\phantomsection\label{\detokenize{QConnectionLibrary:QConnectionLibrary.connection_manager.TestOption}}\pysigline{\sphinxbfcode{\sphinxupquote{class }}\sphinxcode{\sphinxupquote{QConnectionLibrary.connection\_manager.}}\sphinxbfcode{\sphinxupquote{TestOption}}}
\sphinxAtStartPar
Bases: \sphinxcode{\sphinxupquote{object}}
\index{DLT\_OPT (QConnectionLibrary.connection\_manager.TestOption attribute)@\spxentry{DLT\_OPT}\spxextra{QConnectionLibrary.connection\_manager.TestOption attribute}}

\begin{fulllineitems}
\phantomsection\label{\detokenize{QConnectionLibrary:QConnectionLibrary.connection_manager.TestOption.DLT_OPT}}\pysigline{\sphinxbfcode{\sphinxupquote{DLT\_OPT}}\sphinxbfcode{\sphinxupquote{ = 0}}}
\end{fulllineitems}

\index{SERIAL\_OPT (QConnectionLibrary.connection\_manager.TestOption attribute)@\spxentry{SERIAL\_OPT}\spxextra{QConnectionLibrary.connection\_manager.TestOption attribute}}

\begin{fulllineitems}
\phantomsection\label{\detokenize{QConnectionLibrary:QConnectionLibrary.connection_manager.TestOption.SERIAL_OPT}}\pysigline{\sphinxbfcode{\sphinxupquote{SERIAL\_OPT}}\sphinxbfcode{\sphinxupquote{ = 2}}}
\end{fulllineitems}

\index{SSH\_OPT (QConnectionLibrary.connection\_manager.TestOption attribute)@\spxentry{SSH\_OPT}\spxextra{QConnectionLibrary.connection\_manager.TestOption attribute}}

\begin{fulllineitems}
\phantomsection\label{\detokenize{QConnectionLibrary:QConnectionLibrary.connection_manager.TestOption.SSH_OPT}}\pysigline{\sphinxbfcode{\sphinxupquote{SSH\_OPT}}\sphinxbfcode{\sphinxupquote{ = 1}}}
\end{fulllineitems}


\end{fulllineitems}

\index{VerifyParam (class in QConnectionLibrary.connection\_manager)@\spxentry{VerifyParam}\spxextra{class in QConnectionLibrary.connection\_manager}}

\begin{fulllineitems}
\phantomsection\label{\detokenize{QConnectionLibrary:QConnectionLibrary.connection_manager.VerifyParam}}\pysiglinewithargsret{\sphinxbfcode{\sphinxupquote{class }}\sphinxcode{\sphinxupquote{QConnectionLibrary.connection\_manager.}}\sphinxbfcode{\sphinxupquote{VerifyParam}}}{\emph{\DUrole{o}{**}\DUrole{n}{dictionary}}}{}
\sphinxAtStartPar
Bases: {\hyperref[\detokenize{QConnectionLibrary:QConnectionLibrary.connection_manager.InputParam}]{\sphinxcrossref{\sphinxcode{\sphinxupquote{QConnectionLibrary.connection\_manager.InputParam}}}}}

\sphinxAtStartPar
Class for storing parameters for verify action.
\index{conn\_name (QConnectionLibrary.connection\_manager.VerifyParam attribute)@\spxentry{conn\_name}\spxextra{QConnectionLibrary.connection\_manager.VerifyParam attribute}}

\begin{fulllineitems}
\phantomsection\label{\detokenize{QConnectionLibrary:QConnectionLibrary.connection_manager.VerifyParam.conn_name}}\pysigline{\sphinxbfcode{\sphinxupquote{conn\_name}}\sphinxbfcode{\sphinxupquote{ = \textquotesingle{}default\_conn\textquotesingle{}}}}
\end{fulllineitems}

\index{eob\_pattern (QConnectionLibrary.connection\_manager.VerifyParam attribute)@\spxentry{eob\_pattern}\spxextra{QConnectionLibrary.connection\_manager.VerifyParam attribute}}

\begin{fulllineitems}
\phantomsection\label{\detokenize{QConnectionLibrary:QConnectionLibrary.connection_manager.VerifyParam.eob_pattern}}\pysigline{\sphinxbfcode{\sphinxupquote{eob\_pattern}}\sphinxbfcode{\sphinxupquote{ = \textquotesingle{}.*\textquotesingle{}}}}
\end{fulllineitems}

\index{fetch\_block (QConnectionLibrary.connection\_manager.VerifyParam attribute)@\spxentry{fetch\_block}\spxextra{QConnectionLibrary.connection\_manager.VerifyParam attribute}}

\begin{fulllineitems}
\phantomsection\label{\detokenize{QConnectionLibrary:QConnectionLibrary.connection_manager.VerifyParam.fetch_block}}\pysigline{\sphinxbfcode{\sphinxupquote{fetch\_block}}\sphinxbfcode{\sphinxupquote{ = False}}}
\end{fulllineitems}

\index{filter\_pattern (QConnectionLibrary.connection\_manager.VerifyParam attribute)@\spxentry{filter\_pattern}\spxextra{QConnectionLibrary.connection\_manager.VerifyParam attribute}}

\begin{fulllineitems}
\phantomsection\label{\detokenize{QConnectionLibrary:QConnectionLibrary.connection_manager.VerifyParam.filter_pattern}}\pysigline{\sphinxbfcode{\sphinxupquote{filter\_pattern}}\sphinxbfcode{\sphinxupquote{ = \textquotesingle{}.*\textquotesingle{}}}}
\end{fulllineitems}

\index{search\_pattern (QConnectionLibrary.connection\_manager.VerifyParam attribute)@\spxentry{search\_pattern}\spxextra{QConnectionLibrary.connection\_manager.VerifyParam attribute}}

\begin{fulllineitems}
\phantomsection\label{\detokenize{QConnectionLibrary:QConnectionLibrary.connection_manager.VerifyParam.search_pattern}}\pysigline{\sphinxbfcode{\sphinxupquote{search\_pattern}}\sphinxbfcode{\sphinxupquote{ = None}}}
\end{fulllineitems}

\index{send\_cmd (QConnectionLibrary.connection\_manager.VerifyParam attribute)@\spxentry{send\_cmd}\spxextra{QConnectionLibrary.connection\_manager.VerifyParam attribute}}

\begin{fulllineitems}
\phantomsection\label{\detokenize{QConnectionLibrary:QConnectionLibrary.connection_manager.VerifyParam.send_cmd}}\pysigline{\sphinxbfcode{\sphinxupquote{send\_cmd}}\sphinxbfcode{\sphinxupquote{ = \textquotesingle{}\textquotesingle{}}}}
\end{fulllineitems}

\index{timeout (QConnectionLibrary.connection\_manager.VerifyParam attribute)@\spxentry{timeout}\spxextra{QConnectionLibrary.connection\_manager.VerifyParam attribute}}

\begin{fulllineitems}
\phantomsection\label{\detokenize{QConnectionLibrary:QConnectionLibrary.connection_manager.VerifyParam.timeout}}\pysigline{\sphinxbfcode{\sphinxupquote{timeout}}\sphinxbfcode{\sphinxupquote{ = 5}}}
\end{fulllineitems}


\end{fulllineitems}

\phantomsection\label{\detokenize{QConnectionLibrary:module-QConnectionLibrary.connection_base}}\index{module@\spxentry{module}!QConnectionLibrary.connection\_base@\spxentry{QConnectionLibrary.connection\_base}}\index{QConnectionLibrary.connection\_base@\spxentry{QConnectionLibrary.connection\_base}!module@\spxentry{module}}\index{BrokenConnError@\spxentry{BrokenConnError}}

\begin{fulllineitems}
\phantomsection\label{\detokenize{QConnectionLibrary:QConnectionLibrary.connection_base.BrokenConnError}}\pysigline{\sphinxbfcode{\sphinxupquote{exception }}\sphinxcode{\sphinxupquote{QConnectionLibrary.connection\_base.}}\sphinxbfcode{\sphinxupquote{BrokenConnError}}}
\sphinxAtStartPar
Bases: \sphinxcode{\sphinxupquote{Exception}}

\end{fulllineitems}

\index{ConnectionBase (class in QConnectionLibrary.connection\_base)@\spxentry{ConnectionBase}\spxextra{class in QConnectionLibrary.connection\_base}}

\begin{fulllineitems}
\phantomsection\label{\detokenize{QConnectionLibrary:QConnectionLibrary.connection_base.ConnectionBase}}\pysiglinewithargsret{\sphinxbfcode{\sphinxupquote{class }}\sphinxcode{\sphinxupquote{QConnectionLibrary.connection\_base.}}\sphinxbfcode{\sphinxupquote{ConnectionBase}}}{\emph{\DUrole{o}{*}\DUrole{n}{args}}, \emph{\DUrole{o}{**}\DUrole{n}{kwargs}}}{}
\sphinxAtStartPar
Bases: \sphinxcode{\sphinxupquote{object}}

\sphinxAtStartPar
Base class for all connection classes.
\index{MAX\_LEN\_BACKTRACE (QConnectionLibrary.connection\_base.ConnectionBase attribute)@\spxentry{MAX\_LEN\_BACKTRACE}\spxextra{QConnectionLibrary.connection\_base.ConnectionBase attribute}}

\begin{fulllineitems}
\phantomsection\label{\detokenize{QConnectionLibrary:QConnectionLibrary.connection_base.ConnectionBase.MAX_LEN_BACKTRACE}}\pysigline{\sphinxbfcode{\sphinxupquote{MAX\_LEN\_BACKTRACE}}\sphinxbfcode{\sphinxupquote{ = 500}}}
\end{fulllineitems}

\index{RECV\_MSGS\_POLLING\_INTERVAL (QConnectionLibrary.connection\_base.ConnectionBase attribute)@\spxentry{RECV\_MSGS\_POLLING\_INTERVAL}\spxextra{QConnectionLibrary.connection\_base.ConnectionBase attribute}}

\begin{fulllineitems}
\phantomsection\label{\detokenize{QConnectionLibrary:QConnectionLibrary.connection_base.ConnectionBase.RECV_MSGS_POLLING_INTERVAL}}\pysigline{\sphinxbfcode{\sphinxupquote{RECV\_MSGS\_POLLING\_INTERVAL}}\sphinxbfcode{\sphinxupquote{ = 0.005}}}
\end{fulllineitems}

\index{activate\_trace\_queue() (QConnectionLibrary.connection\_base.ConnectionBase class method)@\spxentry{activate\_trace\_queue()}\spxextra{QConnectionLibrary.connection\_base.ConnectionBase class method}}

\begin{fulllineitems}
\phantomsection\label{\detokenize{QConnectionLibrary:QConnectionLibrary.connection_base.ConnectionBase.activate_trace_queue}}\pysiglinewithargsret{\sphinxbfcode{\sphinxupquote{classmethod }}\sphinxbfcode{\sphinxupquote{activate\_trace\_queue}}}{\emph{\DUrole{n}{search\_obj}}, \emph{\DUrole{n}{trace\_queue}}, \emph{\DUrole{n}{use\_fetch\_block}\DUrole{o}{=}\DUrole{default_value}{False}}, \emph{\DUrole{n}{end\_of\_block\_pattern}\DUrole{o}{=}\DUrole{default_value}{\textquotesingle{}.*\textquotesingle{}}}, \emph{\DUrole{n}{line\_filter\_pattern}\DUrole{o}{=}\DUrole{default_value}{None}}}{}
\sphinxAtStartPar
Activates a trace message filter specified as a regular expression. All matching trace messages are put in the specified queue object.
\begin{description}
\item[{Args:}] \leavevmode
\sphinxAtStartPar
search\_obj :  Regular expression all received trace messages are compare to.                        Can be passed either as a string or a regular expression object. Refer to Python documentation for module ‘re’.\#

\sphinxAtStartPar
trace\_queue : A queue object all trace message which matches the regular expression are put in.                        The using application must assure, that the queue is emptied or deleted.

\sphinxAtStartPar
use\_fetch\_block : Determine if ‘fetch block’ feature is used

\sphinxAtStartPar
end\_of\_block\_pattern : The end of block pattern

\sphinxAtStartPar
line\_filter\_pattern : Regular expression object to filter message line by line.

\item[{Returns:}] \leavevmode
\sphinxAtStartPar
\textless{}int\textgreater{} : Handle to deactivate the message filter.

\end{description}

\end{fulllineitems}

\index{check\_timeout() (QConnectionLibrary.connection\_base.ConnectionBase method)@\spxentry{check\_timeout()}\spxextra{QConnectionLibrary.connection\_base.ConnectionBase method}}

\begin{fulllineitems}
\phantomsection\label{\detokenize{QConnectionLibrary:QConnectionLibrary.connection_base.ConnectionBase.check_timeout}}\pysiglinewithargsret{\sphinxbfcode{\sphinxupquote{check\_timeout}}}{\emph{\DUrole{n}{msg}}}{}
\sphinxAtStartPar
\textgreater{}\textgreater{} This method will be override in derived class \textless{}\textless{}
Check if responded message come in cls.\_RESPOND\_TIMEOUT or we will raise a timeout event.
\begin{description}
\item[{Args:}] \leavevmode
\sphinxAtStartPar
msg: Responded message for checking.

\item[{Returns:}] \leavevmode
\sphinxAtStartPar
None.

\end{description}

\end{fulllineitems}

\index{config (QConnectionLibrary.connection\_base.ConnectionBase attribute)@\spxentry{config}\spxextra{QConnectionLibrary.connection\_base.ConnectionBase attribute}}

\begin{fulllineitems}
\phantomsection\label{\detokenize{QConnectionLibrary:QConnectionLibrary.connection_base.ConnectionBase.config}}\pysigline{\sphinxbfcode{\sphinxupquote{config}}\sphinxbfcode{\sphinxupquote{ = None}}}
\end{fulllineitems}

\index{connect() (QConnectionLibrary.connection\_base.ConnectionBase method)@\spxentry{connect()}\spxextra{QConnectionLibrary.connection\_base.ConnectionBase method}}

\begin{fulllineitems}
\phantomsection\label{\detokenize{QConnectionLibrary:QConnectionLibrary.connection_base.ConnectionBase.connect}}\pysiglinewithargsret{\sphinxbfcode{\sphinxupquote{abstract }}\sphinxbfcode{\sphinxupquote{connect}}}{\emph{\DUrole{n}{device}}, \emph{\DUrole{n}{files}\DUrole{o}{=}\DUrole{default_value}{None}}, \emph{\DUrole{n}{test\_connection}\DUrole{o}{=}\DUrole{default_value}{False}}}{}
\sphinxAtStartPar
\textgreater{}\textgreater{} This method MUST be overridden in derived class \textless{}\textless{}
Abstract method for quiting the connection.
\begin{description}
\item[{Args:}] \leavevmode
\sphinxAtStartPar
device: Determine if it’s necessary to disconnect all connections.

\sphinxAtStartPar
files: Determine if it’s necessary to disconnect all connections.

\sphinxAtStartPar
test\_connection: Determine if it’s necessary to disconnect all connections.

\item[{Returns:}] \leavevmode
\sphinxAtStartPar
None.

\end{description}

\end{fulllineitems}

\index{create\_and\_activate\_trace\_queue() (QConnectionLibrary.connection\_base.ConnectionBase class method)@\spxentry{create\_and\_activate\_trace\_queue()}\spxextra{QConnectionLibrary.connection\_base.ConnectionBase class method}}

\begin{fulllineitems}
\phantomsection\label{\detokenize{QConnectionLibrary:QConnectionLibrary.connection_base.ConnectionBase.create_and_activate_trace_queue}}\pysiglinewithargsret{\sphinxbfcode{\sphinxupquote{classmethod }}\sphinxbfcode{\sphinxupquote{create\_and\_activate\_trace\_queue}}}{\emph{\DUrole{n}{search\_element}}, \emph{\DUrole{n}{use\_fetch\_block}\DUrole{o}{=}\DUrole{default_value}{False}}, \emph{\DUrole{n}{end\_of\_block\_pattern}\DUrole{o}{=}\DUrole{default_value}{\textquotesingle{}.*\textquotesingle{}}}, \emph{\DUrole{n}{regex\_line\_filter\_pattern}\DUrole{o}{=}\DUrole{default_value}{None}}}{}
\sphinxAtStartPar
Create Queue and assign it to \_trace\_queue object and activate the queue with the search element.
\begin{description}
\item[{Args:}] \leavevmode
\sphinxAtStartPar
search\_element :  Regular expression all received trace messages are compare to.                Can be passed either as a string or a regular expression object. Refer to Python documentation for module ‘re’.\#

\sphinxAtStartPar
use\_fetch\_block : Determine if ‘fetch block’ feature is used.

\sphinxAtStartPar
end\_of\_block\_pattern : The end of block pattern.

\sphinxAtStartPar
regex\_line\_filter\_pattern : Regular expression object to filter message line by line.

\item[{Returns:}] \leavevmode
\sphinxAtStartPar
trq\_handle, trace\_queue: the handle and search object

\end{description}

\end{fulllineitems}

\index{deactivate\_and\_delete\_trace\_queue() (QConnectionLibrary.connection\_base.ConnectionBase class method)@\spxentry{deactivate\_and\_delete\_trace\_queue()}\spxextra{QConnectionLibrary.connection\_base.ConnectionBase class method}}

\begin{fulllineitems}
\phantomsection\label{\detokenize{QConnectionLibrary:QConnectionLibrary.connection_base.ConnectionBase.deactivate_and_delete_trace_queue}}\pysiglinewithargsret{\sphinxbfcode{\sphinxupquote{classmethod }}\sphinxbfcode{\sphinxupquote{deactivate\_and\_delete\_trace\_queue}}}{\emph{\DUrole{n}{trq\_handle}}, \emph{\DUrole{n}{trace\_queue}}}{}
\sphinxAtStartPar
Deactivate trace queue and delete.
\begin{description}
\item[{Args:}] \leavevmode
\sphinxAtStartPar
trq\_handle: Trace queue handle.

\sphinxAtStartPar
trace\_queue: Trace queue object.

\item[{Returns:}] \leavevmode
\sphinxAtStartPar
None.

\end{description}

\end{fulllineitems}

\index{deactivate\_trace\_queue() (QConnectionLibrary.connection\_base.ConnectionBase class method)@\spxentry{deactivate\_trace\_queue()}\spxextra{QConnectionLibrary.connection\_base.ConnectionBase class method}}

\begin{fulllineitems}
\phantomsection\label{\detokenize{QConnectionLibrary:QConnectionLibrary.connection_base.ConnectionBase.deactivate_trace_queue}}\pysiglinewithargsret{\sphinxbfcode{\sphinxupquote{classmethod }}\sphinxbfcode{\sphinxupquote{deactivate\_trace\_queue}}}{\emph{\DUrole{n}{handle}}}{}
\sphinxAtStartPar
Deactivates a trace message filter previously activated by ActivateTraceQ() method.
\begin{description}
\item[{Args:}] \leavevmode
\sphinxAtStartPar
handle :  Integer object returned by ActivateTraceQ() method.

\item[{Returns:}] \leavevmode
\sphinxAtStartPar
False : No trace message filter active with the specified handle (i.e. handle is not in use).

\sphinxAtStartPar
True :  Trace message filter successfully deleted.

\end{description}

\end{fulllineitems}

\index{disconnect() (QConnectionLibrary.connection\_base.ConnectionBase method)@\spxentry{disconnect()}\spxextra{QConnectionLibrary.connection\_base.ConnectionBase method}}

\begin{fulllineitems}
\phantomsection\label{\detokenize{QConnectionLibrary:QConnectionLibrary.connection_base.ConnectionBase.disconnect}}\pysiglinewithargsret{\sphinxbfcode{\sphinxupquote{abstract }}\sphinxbfcode{\sphinxupquote{disconnect}}}{\emph{\DUrole{n}{device}}}{}
\sphinxAtStartPar
\textgreater{}\textgreater{} This method MUST be overridden in derived class \textless{}\textless{}
Abstract method for disconnecting connection.
\begin{description}
\item[{Args:}] \leavevmode
\sphinxAtStartPar
device: Device name.

\item[{Returns:}] \leavevmode
\sphinxAtStartPar
None.

\end{description}

\end{fulllineitems}

\index{error\_instruction() (QConnectionLibrary.connection\_base.ConnectionBase method)@\spxentry{error\_instruction()}\spxextra{QConnectionLibrary.connection\_base.ConnectionBase method}}

\begin{fulllineitems}
\phantomsection\label{\detokenize{QConnectionLibrary:QConnectionLibrary.connection_base.ConnectionBase.error_instruction}}\pysiglinewithargsret{\sphinxbfcode{\sphinxupquote{error\_instruction}}}{}{}
\sphinxAtStartPar
Get the error instruction.
\begin{description}
\item[{Returns:}] \leavevmode
\sphinxAtStartPar
Error instruction string.

\end{description}

\end{fulllineitems}

\index{is\_precondition\_pass() (QConnectionLibrary.connection\_base.ConnectionBase class method)@\spxentry{is\_precondition\_pass()}\spxextra{QConnectionLibrary.connection\_base.ConnectionBase class method}}

\begin{fulllineitems}
\phantomsection\label{\detokenize{QConnectionLibrary:QConnectionLibrary.connection_base.ConnectionBase.is_precondition_pass}}\pysiglinewithargsret{\sphinxbfcode{\sphinxupquote{classmethod }}\sphinxbfcode{\sphinxupquote{is\_precondition\_pass}}}{}{}
\sphinxAtStartPar
Check for precondition.
\begin{description}
\item[{Returns:}] \leavevmode
\sphinxAtStartPar
True if passing the precondition.

\sphinxAtStartPar
False if failing the precondition.

\end{description}

\end{fulllineitems}

\index{is\_supported\_platform() (QConnectionLibrary.connection\_base.ConnectionBase class method)@\spxentry{is\_supported\_platform()}\spxextra{QConnectionLibrary.connection\_base.ConnectionBase class method}}

\begin{fulllineitems}
\phantomsection\label{\detokenize{QConnectionLibrary:QConnectionLibrary.connection_base.ConnectionBase.is_supported_platform}}\pysiglinewithargsret{\sphinxbfcode{\sphinxupquote{classmethod }}\sphinxbfcode{\sphinxupquote{is\_supported\_platform}}}{}{}
\sphinxAtStartPar
Check if current platform is supported.
\begin{description}
\item[{Returns:}] \leavevmode
\sphinxAtStartPar
True if platform is supported.

\sphinxAtStartPar
False if platform is not supported.

\end{description}

\end{fulllineitems}

\index{post\_msg\_check() (QConnectionLibrary.connection\_base.ConnectionBase method)@\spxentry{post\_msg\_check()}\spxextra{QConnectionLibrary.connection\_base.ConnectionBase method}}

\begin{fulllineitems}
\phantomsection\label{\detokenize{QConnectionLibrary:QConnectionLibrary.connection_base.ConnectionBase.post_msg_check}}\pysiglinewithargsret{\sphinxbfcode{\sphinxupquote{post\_msg\_check}}}{\emph{\DUrole{n}{msg}}}{}
\sphinxAtStartPar
\textgreater{}\textgreater{} This method will be override in derived class \textless{}\textless{}
Post\sphinxhyphen{}checking message when receiving it from connection.
\begin{description}
\item[{Args:}] \leavevmode
\sphinxAtStartPar
msg: received message to be checked.

\item[{Returns:}] \leavevmode
\sphinxAtStartPar
None.

\end{description}

\end{fulllineitems}

\index{pre\_msg\_check() (QConnectionLibrary.connection\_base.ConnectionBase method)@\spxentry{pre\_msg\_check()}\spxextra{QConnectionLibrary.connection\_base.ConnectionBase method}}

\begin{fulllineitems}
\phantomsection\label{\detokenize{QConnectionLibrary:QConnectionLibrary.connection_base.ConnectionBase.pre_msg_check}}\pysiglinewithargsret{\sphinxbfcode{\sphinxupquote{pre\_msg\_check}}}{\emph{\DUrole{n}{msg}}}{}
\sphinxAtStartPar
\textgreater{}\textgreater{} This method will be override in derived class \textless{}\textless{}
Pre\sphinxhyphen{}checking message when receiving it from connection.
\begin{description}
\item[{Args:}] \leavevmode
\sphinxAtStartPar
msg: received message to be checked.

\item[{Returns:}] \leavevmode
\sphinxAtStartPar
None.

\end{description}

\end{fulllineitems}

\index{quit() (QConnectionLibrary.connection\_base.ConnectionBase method)@\spxentry{quit()}\spxextra{QConnectionLibrary.connection\_base.ConnectionBase method}}

\begin{fulllineitems}
\phantomsection\label{\detokenize{QConnectionLibrary:QConnectionLibrary.connection_base.ConnectionBase.quit}}\pysiglinewithargsret{\sphinxbfcode{\sphinxupquote{abstract }}\sphinxbfcode{\sphinxupquote{quit}}}{\emph{\DUrole{n}{is\_disconnect\_all}\DUrole{o}{=}\DUrole{default_value}{True}}}{}
\sphinxAtStartPar
\textgreater{}\textgreater{} This method MUST be overridden in derived class \textless{}\textless{}
Abstract method for quiting the connection.
\begin{description}
\item[{Args:}] \leavevmode
\sphinxAtStartPar
is\_disconnect\_all: Determine if it’s necessary to disconnect all connections.

\item[{Returns:}] \leavevmode
\sphinxAtStartPar
None.

\end{description}

\end{fulllineitems}

\index{read\_obj() (QConnectionLibrary.connection\_base.ConnectionBase method)@\spxentry{read\_obj()}\spxextra{QConnectionLibrary.connection\_base.ConnectionBase method}}

\begin{fulllineitems}
\phantomsection\label{\detokenize{QConnectionLibrary:QConnectionLibrary.connection_base.ConnectionBase.read_obj}}\pysiglinewithargsret{\sphinxbfcode{\sphinxupquote{read\_obj}}}{}{}
\sphinxAtStartPar
Wrapper method to get the response from connection.
\begin{description}
\item[{Returns:}] \leavevmode
\sphinxAtStartPar
Responded message.

\end{description}

\end{fulllineitems}

\index{send\_obj() (QConnectionLibrary.connection\_base.ConnectionBase method)@\spxentry{send\_obj()}\spxextra{QConnectionLibrary.connection\_base.ConnectionBase method}}

\begin{fulllineitems}
\phantomsection\label{\detokenize{QConnectionLibrary:QConnectionLibrary.connection_base.ConnectionBase.send_obj}}\pysiglinewithargsret{\sphinxbfcode{\sphinxupquote{send\_obj}}}{\emph{\DUrole{n}{obj}}, \emph{\DUrole{n}{cr}\DUrole{o}{=}\DUrole{default_value}{True}}}{}
\sphinxAtStartPar
Wrapper method to send message to a tcp connection.
\begin{description}
\item[{Args:}] \leavevmode
\sphinxAtStartPar
obj: Data to be sent.
cr: Determine if it’s necessary to add newline character at the end of command.

\item[{Returns:}] \leavevmode
\sphinxAtStartPar
None

\end{description}

\end{fulllineitems}

\index{supported\_devices (QConnectionLibrary.connection\_base.ConnectionBase attribute)@\spxentry{supported\_devices}\spxextra{QConnectionLibrary.connection\_base.ConnectionBase attribute}}

\begin{fulllineitems}
\phantomsection\label{\detokenize{QConnectionLibrary:QConnectionLibrary.connection_base.ConnectionBase.supported_devices}}\pysigline{\sphinxbfcode{\sphinxupquote{supported\_devices}}\sphinxbfcode{\sphinxupquote{ = {[}{]}}}}
\end{fulllineitems}

\index{wait\_4\_trace() (QConnectionLibrary.connection\_base.ConnectionBase method)@\spxentry{wait\_4\_trace()}\spxextra{QConnectionLibrary.connection\_base.ConnectionBase method}}

\begin{fulllineitems}
\phantomsection\label{\detokenize{QConnectionLibrary:QConnectionLibrary.connection_base.ConnectionBase.wait_4_trace}}\pysiglinewithargsret{\sphinxbfcode{\sphinxupquote{wait\_4\_trace}}}{\emph{\DUrole{n}{search\_obj}}, \emph{\DUrole{n}{timeout}\DUrole{o}{=}\DUrole{default_value}{0}}, \emph{\DUrole{n}{use\_fetch\_block}\DUrole{o}{=}\DUrole{default_value}{False}}, \emph{\DUrole{n}{end\_of\_block\_pattern}\DUrole{o}{=}\DUrole{default_value}{\textquotesingle{}.*\textquotesingle{}}}, \emph{\DUrole{n}{filter\_pattern}\DUrole{o}{=}\DUrole{default_value}{\textquotesingle{}.*\textquotesingle{}}}, \emph{\DUrole{o}{*}\DUrole{n}{fct\_args}}}{}
\sphinxAtStartPar
Suspend the control flow until a Trace message is received which matches to a specified regular expression.
\begin{description}
\item[{Args:}] \leavevmode
\sphinxAtStartPar
search\_obj : Regular expression all received trace messages are compare to.           Can be passed either as a string or a regular expression object. Refer to Python documentation for module ‘re’.

\sphinxAtStartPar
use\_fetch\_block : Determine if ‘fetch block’ feature is used.

\sphinxAtStartPar
end\_of\_block\_pattern : The end of block pattern.

\sphinxAtStartPar
filter\_pattern : Regular expression object to filter message line by line.

\sphinxAtStartPar
timeout :   Optional timeout parameter specified as a floating point number in the unit ‘seconds’.

\sphinxAtStartPar
fct\_args:   Optional list of function arguments passed to be sent.

\item[{Returns:}] \leavevmode
\sphinxAtStartPar
None :    If no trace message matched to the specified regular expression and a timeout occurred.

\sphinxAtStartPar
\textless{}match\textgreater{} : If a trace message has matched to the specified regular expression, a match object is returned as the result.                    The complete trace message can be accessed by the ‘string’ attribute of the match object.                    For access to groups within the regular expression, use the group() method.                    For more information, refer to Python documentation for module ‘re’.

\end{description}

\end{fulllineitems}

\index{wait\_4\_trace\_continuously() (QConnectionLibrary.connection\_base.ConnectionBase method)@\spxentry{wait\_4\_trace\_continuously()}\spxextra{QConnectionLibrary.connection\_base.ConnectionBase method}}

\begin{fulllineitems}
\phantomsection\label{\detokenize{QConnectionLibrary:QConnectionLibrary.connection_base.ConnectionBase.wait_4_trace_continuously}}\pysiglinewithargsret{\sphinxbfcode{\sphinxupquote{wait\_4\_trace\_continuously}}}{\emph{\DUrole{n}{trace\_queue}}, \emph{\DUrole{n}{timeout}\DUrole{o}{=}\DUrole{default_value}{0}}, \emph{\DUrole{o}{*}\DUrole{n}{fct\_args}}}{}
\sphinxAtStartPar
Getting trace log continuously without creating a new trace queue.
\begin{description}
\item[{Args:}] \leavevmode
\sphinxAtStartPar
trace\_queue: Queue to store the traces.

\sphinxAtStartPar
timeout: Timeout for waiting a matched log.

\sphinxAtStartPar
fct\_args: Arguments to be sent to connection.

\item[{Returns:}] \leavevmode
\sphinxAtStartPar
None :    If no trace message matched to the specified regular expression and a timeout occurred.

\sphinxAtStartPar
match object : If a trace message has matched to the specified regular expression, a match object is returned as the result.                         The complete trace message can be accessed by the ‘string’ attribute of the match object.                         For access to groups within the regular expression, use the group() method.                         For more information, refer to Python documentation for module ‘re’.

\end{description}

\end{fulllineitems}


\end{fulllineitems}

\phantomsection\label{\detokenize{QConnectionLibrary:module-QConnectionLibrary.qlogger}}\index{module@\spxentry{module}!QConnectionLibrary.qlogger@\spxentry{QConnectionLibrary.qlogger}}\index{QConnectionLibrary.qlogger@\spxentry{QConnectionLibrary.qlogger}!module@\spxentry{module}}\index{ColorFormatter (class in QConnectionLibrary.qlogger)@\spxentry{ColorFormatter}\spxextra{class in QConnectionLibrary.qlogger}}

\begin{fulllineitems}
\phantomsection\label{\detokenize{QConnectionLibrary:QConnectionLibrary.qlogger.ColorFormatter}}\pysiglinewithargsret{\sphinxbfcode{\sphinxupquote{class }}\sphinxcode{\sphinxupquote{QConnectionLibrary.qlogger.}}\sphinxbfcode{\sphinxupquote{ColorFormatter}}}{\emph{\DUrole{n}{fmt}\DUrole{o}{=}\DUrole{default_value}{None}}, \emph{\DUrole{n}{datefmt}\DUrole{o}{=}\DUrole{default_value}{None}}, \emph{\DUrole{n}{style}\DUrole{o}{=}\DUrole{default_value}{\textquotesingle{}\%\textquotesingle{}}}, \emph{\DUrole{n}{validate}\DUrole{o}{=}\DUrole{default_value}{True}}}{}
\sphinxAtStartPar
Bases: \sphinxcode{\sphinxupquote{logging.Formatter}}

\sphinxAtStartPar
Custom formatter class for setting log color.
\index{FORMATS (QConnectionLibrary.qlogger.ColorFormatter attribute)@\spxentry{FORMATS}\spxextra{QConnectionLibrary.qlogger.ColorFormatter attribute}}

\begin{fulllineitems}
\phantomsection\label{\detokenize{QConnectionLibrary:QConnectionLibrary.qlogger.ColorFormatter.FORMATS}}\pysigline{\sphinxbfcode{\sphinxupquote{FORMATS}}\sphinxbfcode{\sphinxupquote{ = \{10: \textquotesingle{}\textbackslash{}x1b{[}38;21m\%(asctime)s {[}\%(threadName)\sphinxhyphen{}12.12s{]} {[}\%(levelname)\sphinxhyphen{}5.5s{]}  \%(message)s\textbackslash{}x1b{[}0m\textquotesingle{}, 20: \textquotesingle{}\textbackslash{}x1b{[}38;21m\%(asctime)s {[}\%(threadName)\sphinxhyphen{}12.12s{]} {[}\%(levelname)\sphinxhyphen{}5.5s{]}  \%(message)s\textbackslash{}x1b{[}0m\textquotesingle{}, 30: \textquotesingle{}\textbackslash{}x1b{[}33;21m\%(asctime)s {[}\%(threadName)\sphinxhyphen{}12.12s{]} {[}\%(levelname)\sphinxhyphen{}5.5s{]}  \%(message)s\textbackslash{}x1b{[}0m\textquotesingle{}, 40: \textquotesingle{}\textbackslash{}x1b{[}31;21m\%(asctime)s {[}\%(threadName)\sphinxhyphen{}12.12s{]} {[}\%(levelname)\sphinxhyphen{}5.5s{]}  \%(message)s\textbackslash{}x1b{[}0m\textquotesingle{}, 50: \textquotesingle{}\textbackslash{}x1b{[}31;1m\%(asctime)s {[}\%(threadName)\sphinxhyphen{}12.12s{]} {[}\%(levelname)\sphinxhyphen{}5.5s{]}  \%(message)s\textbackslash{}x1b{[}0m\textquotesingle{}\}}}}
\end{fulllineitems}

\index{bold\_red (QConnectionLibrary.qlogger.ColorFormatter attribute)@\spxentry{bold\_red}\spxextra{QConnectionLibrary.qlogger.ColorFormatter attribute}}

\begin{fulllineitems}
\phantomsection\label{\detokenize{QConnectionLibrary:QConnectionLibrary.qlogger.ColorFormatter.bold_red}}\pysigline{\sphinxbfcode{\sphinxupquote{bold\_red}}\sphinxbfcode{\sphinxupquote{ = \textquotesingle{}\textbackslash{}x1b{[}31;1m\textquotesingle{}}}}
\end{fulllineitems}

\index{format() (QConnectionLibrary.qlogger.ColorFormatter method)@\spxentry{format()}\spxextra{QConnectionLibrary.qlogger.ColorFormatter method}}

\begin{fulllineitems}
\phantomsection\label{\detokenize{QConnectionLibrary:QConnectionLibrary.qlogger.ColorFormatter.format}}\pysiglinewithargsret{\sphinxbfcode{\sphinxupquote{format}}}{\emph{\DUrole{n}{record}}}{}
\sphinxAtStartPar
Set the color format for the log.
\begin{description}
\item[{Args:}] \leavevmode
\sphinxAtStartPar
record: log record.

\item[{Returns:}] \leavevmode
\sphinxAtStartPar
Log with color formatter.

\end{description}

\end{fulllineitems}

\index{grey (QConnectionLibrary.qlogger.ColorFormatter attribute)@\spxentry{grey}\spxextra{QConnectionLibrary.qlogger.ColorFormatter attribute}}

\begin{fulllineitems}
\phantomsection\label{\detokenize{QConnectionLibrary:QConnectionLibrary.qlogger.ColorFormatter.grey}}\pysigline{\sphinxbfcode{\sphinxupquote{grey}}\sphinxbfcode{\sphinxupquote{ = \textquotesingle{}\textbackslash{}x1b{[}38;21m\textquotesingle{}}}}
\end{fulllineitems}

\index{red (QConnectionLibrary.qlogger.ColorFormatter attribute)@\spxentry{red}\spxextra{QConnectionLibrary.qlogger.ColorFormatter attribute}}

\begin{fulllineitems}
\phantomsection\label{\detokenize{QConnectionLibrary:QConnectionLibrary.qlogger.ColorFormatter.red}}\pysigline{\sphinxbfcode{\sphinxupquote{red}}\sphinxbfcode{\sphinxupquote{ = \textquotesingle{}\textbackslash{}x1b{[}31;21m\textquotesingle{}}}}
\end{fulllineitems}

\index{reset (QConnectionLibrary.qlogger.ColorFormatter attribute)@\spxentry{reset}\spxextra{QConnectionLibrary.qlogger.ColorFormatter attribute}}

\begin{fulllineitems}
\phantomsection\label{\detokenize{QConnectionLibrary:QConnectionLibrary.qlogger.ColorFormatter.reset}}\pysigline{\sphinxbfcode{\sphinxupquote{reset}}\sphinxbfcode{\sphinxupquote{ = \textquotesingle{}\textbackslash{}x1b{[}0m\textquotesingle{}}}}
\end{fulllineitems}

\index{yellow (QConnectionLibrary.qlogger.ColorFormatter attribute)@\spxentry{yellow}\spxextra{QConnectionLibrary.qlogger.ColorFormatter attribute}}

\begin{fulllineitems}
\phantomsection\label{\detokenize{QConnectionLibrary:QConnectionLibrary.qlogger.ColorFormatter.yellow}}\pysigline{\sphinxbfcode{\sphinxupquote{yellow}}\sphinxbfcode{\sphinxupquote{ = \textquotesingle{}\textbackslash{}x1b{[}33;21m\textquotesingle{}}}}
\end{fulllineitems}


\end{fulllineitems}

\index{QConsoleHandler (class in QConnectionLibrary.qlogger)@\spxentry{QConsoleHandler}\spxextra{class in QConnectionLibrary.qlogger}}

\begin{fulllineitems}
\phantomsection\label{\detokenize{QConnectionLibrary:QConnectionLibrary.qlogger.QConsoleHandler}}\pysiglinewithargsret{\sphinxbfcode{\sphinxupquote{class }}\sphinxcode{\sphinxupquote{QConnectionLibrary.qlogger.}}\sphinxbfcode{\sphinxupquote{QConsoleHandler}}}{\emph{\DUrole{n}{\_config}}, \emph{\DUrole{n}{\_logger\_name}}, \emph{\DUrole{n}{\_formatter}}}{}
\sphinxAtStartPar
Bases: \sphinxcode{\sphinxupquote{logging.StreamHandler}}

\sphinxAtStartPar
Handler class for console log.
\index{get\_config\_supported() (QConnectionLibrary.qlogger.QConsoleHandler static method)@\spxentry{get\_config\_supported()}\spxextra{QConnectionLibrary.qlogger.QConsoleHandler static method}}

\begin{fulllineitems}
\phantomsection\label{\detokenize{QConnectionLibrary:QConnectionLibrary.qlogger.QConsoleHandler.get_config_supported}}\pysiglinewithargsret{\sphinxbfcode{\sphinxupquote{static }}\sphinxbfcode{\sphinxupquote{get\_config\_supported}}}{\emph{\DUrole{n}{config}}}{}
\sphinxAtStartPar
Check if the connection config is supported by this handler.
\begin{description}
\item[{Args:}] \leavevmode
\sphinxAtStartPar
config: connection configurations.

\item[{Returns:}] \leavevmode
\sphinxAtStartPar
True if the config is supported.

\sphinxAtStartPar
False if the config is not supported.

\end{description}

\end{fulllineitems}


\end{fulllineitems}

\index{QDefaultFileHandler (class in QConnectionLibrary.qlogger)@\spxentry{QDefaultFileHandler}\spxextra{class in QConnectionLibrary.qlogger}}

\begin{fulllineitems}
\phantomsection\label{\detokenize{QConnectionLibrary:QConnectionLibrary.qlogger.QDefaultFileHandler}}\pysiglinewithargsret{\sphinxbfcode{\sphinxupquote{class }}\sphinxcode{\sphinxupquote{QConnectionLibrary.qlogger.}}\sphinxbfcode{\sphinxupquote{QDefaultFileHandler}}}{\emph{\DUrole{n}{\_config}}, \emph{\DUrole{n}{logger\_name}}, \emph{\DUrole{n}{formatter}}}{}
\sphinxAtStartPar
Bases: \sphinxcode{\sphinxupquote{logging.FileHandler}}

\sphinxAtStartPar
Handler class for default log file path.
\index{get\_config\_supported() (QConnectionLibrary.qlogger.QDefaultFileHandler static method)@\spxentry{get\_config\_supported()}\spxextra{QConnectionLibrary.qlogger.QDefaultFileHandler static method}}

\begin{fulllineitems}
\phantomsection\label{\detokenize{QConnectionLibrary:QConnectionLibrary.qlogger.QDefaultFileHandler.get_config_supported}}\pysiglinewithargsret{\sphinxbfcode{\sphinxupquote{static }}\sphinxbfcode{\sphinxupquote{get\_config\_supported}}}{\emph{\DUrole{n}{config}}}{}
\sphinxAtStartPar
Check if the connection config is supported by this handler.
\begin{description}
\item[{Args:}] \leavevmode
\sphinxAtStartPar
config: connection configurations.

\item[{Returns:}] \leavevmode
\sphinxAtStartPar
True if the config is supported.

\sphinxAtStartPar
False if the config is not supported.

\end{description}

\end{fulllineitems}

\index{get\_log\_path() (QConnectionLibrary.qlogger.QDefaultFileHandler static method)@\spxentry{get\_log\_path()}\spxextra{QConnectionLibrary.qlogger.QDefaultFileHandler static method}}

\begin{fulllineitems}
\phantomsection\label{\detokenize{QConnectionLibrary:QConnectionLibrary.qlogger.QDefaultFileHandler.get_log_path}}\pysiglinewithargsret{\sphinxbfcode{\sphinxupquote{static }}\sphinxbfcode{\sphinxupquote{get\_log\_path}}}{\emph{\DUrole{n}{logger\_name}}}{}
\sphinxAtStartPar
Get the log file path for this handler.
\begin{description}
\item[{Args:}] \leavevmode
\sphinxAtStartPar
logger\_name: name of the logger.

\item[{Returns:}] \leavevmode
\sphinxAtStartPar
Log file path.

\end{description}

\end{fulllineitems}


\end{fulllineitems}

\index{QFileHandler (class in QConnectionLibrary.qlogger)@\spxentry{QFileHandler}\spxextra{class in QConnectionLibrary.qlogger}}

\begin{fulllineitems}
\phantomsection\label{\detokenize{QConnectionLibrary:QConnectionLibrary.qlogger.QFileHandler}}\pysiglinewithargsret{\sphinxbfcode{\sphinxupquote{class }}\sphinxcode{\sphinxupquote{QConnectionLibrary.qlogger.}}\sphinxbfcode{\sphinxupquote{QFileHandler}}}{\emph{\DUrole{n}{config}}, \emph{\DUrole{n}{\_logger\_name}}, \emph{\DUrole{n}{formatter}}}{}
\sphinxAtStartPar
Bases: \sphinxcode{\sphinxupquote{logging.FileHandler}}

\sphinxAtStartPar
Handler class for user defined file in config.
\index{get\_config\_supported() (QConnectionLibrary.qlogger.QFileHandler static method)@\spxentry{get\_config\_supported()}\spxextra{QConnectionLibrary.qlogger.QFileHandler static method}}

\begin{fulllineitems}
\phantomsection\label{\detokenize{QConnectionLibrary:QConnectionLibrary.qlogger.QFileHandler.get_config_supported}}\pysiglinewithargsret{\sphinxbfcode{\sphinxupquote{static }}\sphinxbfcode{\sphinxupquote{get\_config\_supported}}}{\emph{\DUrole{n}{config}}}{}
\sphinxAtStartPar
Check if the connection config is supported by this handler.
\begin{description}
\item[{Args:}] \leavevmode
\sphinxAtStartPar
config: connection configurations.

\item[{Returns:}] \leavevmode
\sphinxAtStartPar
True if the config is supported.

\sphinxAtStartPar
False if the config is not supported.

\end{description}

\end{fulllineitems}

\index{get\_log\_path() (QConnectionLibrary.qlogger.QFileHandler static method)@\spxentry{get\_log\_path()}\spxextra{QConnectionLibrary.qlogger.QFileHandler static method}}

\begin{fulllineitems}
\phantomsection\label{\detokenize{QConnectionLibrary:QConnectionLibrary.qlogger.QFileHandler.get_log_path}}\pysiglinewithargsret{\sphinxbfcode{\sphinxupquote{static }}\sphinxbfcode{\sphinxupquote{get\_log\_path}}}{\emph{\DUrole{n}{config}}}{}
\sphinxAtStartPar
Get the log file path for this handler.
\begin{description}
\item[{Args:}] \leavevmode
\sphinxAtStartPar
config: connection configurations.

\item[{Returns:}] \leavevmode
\sphinxAtStartPar
Log file path.

\end{description}

\end{fulllineitems}


\end{fulllineitems}

\index{QLogger (class in QConnectionLibrary.qlogger)@\spxentry{QLogger}\spxextra{class in QConnectionLibrary.qlogger}}

\begin{fulllineitems}
\phantomsection\label{\detokenize{QConnectionLibrary:QConnectionLibrary.qlogger.QLogger}}\pysiglinewithargsret{\sphinxbfcode{\sphinxupquote{class }}\sphinxcode{\sphinxupquote{QConnectionLibrary.qlogger.}}\sphinxbfcode{\sphinxupquote{QLogger}}}{\emph{\DUrole{o}{*}\DUrole{n}{args}}, \emph{\DUrole{o}{**}\DUrole{n}{kwargs}}}{}
\sphinxAtStartPar
Bases: \sphinxcode{\sphinxupquote{QConnectionLibrary.utils.Singleton}}

\sphinxAtStartPar
Logger class for QConnect Libraries.
\index{NAME\_2\_LEVEL\_DICT (QConnectionLibrary.qlogger.QLogger attribute)@\spxentry{NAME\_2\_LEVEL\_DICT}\spxextra{QConnectionLibrary.qlogger.QLogger attribute}}

\begin{fulllineitems}
\phantomsection\label{\detokenize{QConnectionLibrary:QConnectionLibrary.qlogger.QLogger.NAME_2_LEVEL_DICT}}\pysigline{\sphinxbfcode{\sphinxupquote{NAME\_2\_LEVEL\_DICT}}\sphinxbfcode{\sphinxupquote{ = \{\textquotesingle{}NONE\textquotesingle{}: 51, \textquotesingle{}TRACE\textquotesingle{}: 0\}}}}
\end{fulllineitems}

\index{get\_logger() (QConnectionLibrary.qlogger.QLogger method)@\spxentry{get\_logger()}\spxextra{QConnectionLibrary.qlogger.QLogger method}}

\begin{fulllineitems}
\phantomsection\label{\detokenize{QConnectionLibrary:QConnectionLibrary.qlogger.QLogger.get_logger}}\pysiglinewithargsret{\sphinxbfcode{\sphinxupquote{get\_logger}}}{\emph{\DUrole{n}{logger\_name}}}{}
\sphinxAtStartPar
Get the logger object.
\begin{description}
\item[{Args:}] \leavevmode
\sphinxAtStartPar
logger\_name: Name of the logger.

\item[{Returns:}] \leavevmode
\sphinxAtStartPar
Logger object.

\end{description}

\end{fulllineitems}

\index{set\_handler() (QConnectionLibrary.qlogger.QLogger method)@\spxentry{set\_handler()}\spxextra{QConnectionLibrary.qlogger.QLogger method}}

\begin{fulllineitems}
\phantomsection\label{\detokenize{QConnectionLibrary:QConnectionLibrary.qlogger.QLogger.set_handler}}\pysiglinewithargsret{\sphinxbfcode{\sphinxupquote{set\_handler}}}{\emph{\DUrole{n}{config}}}{}
\sphinxAtStartPar
Set handler for logger.
\begin{description}
\item[{Args:}] \leavevmode
\sphinxAtStartPar
config: connection configurations.

\item[{Returns:}] \leavevmode
\sphinxAtStartPar
None if no handler is set.
Handler object.

\end{description}

\end{fulllineitems}


\end{fulllineitems}

\phantomsection\label{\detokenize{QConnectionLibrary:module-QConnectionLibrary.tcp.tcp_base}}\index{module@\spxentry{module}!QConnectionLibrary.tcp.tcp\_base@\spxentry{QConnectionLibrary.tcp.tcp\_base}}\index{QConnectionLibrary.tcp.tcp\_base@\spxentry{QConnectionLibrary.tcp.tcp\_base}!module@\spxentry{module}}\index{TCPBase (class in QConnectionLibrary.tcp.tcp\_base)@\spxentry{TCPBase}\spxextra{class in QConnectionLibrary.tcp.tcp\_base}}

\begin{fulllineitems}
\phantomsection\label{\detokenize{QConnectionLibrary:QConnectionLibrary.tcp.tcp_base.TCPBase}}\pysiglinewithargsret{\sphinxbfcode{\sphinxupquote{class }}\sphinxcode{\sphinxupquote{QConnectionLibrary.tcp.tcp\_base.}}\sphinxbfcode{\sphinxupquote{TCPBase}}}{\emph{\DUrole{o}{*}\DUrole{n}{args}}, \emph{\DUrole{o}{**}\DUrole{n}{kwargs}}}{}
\sphinxAtStartPar
Bases: {\hyperref[\detokenize{QConnectionLibrary:QConnectionLibrary.connection_base.ConnectionBase}]{\sphinxcrossref{\sphinxcode{\sphinxupquote{QConnectionLibrary.connection\_base.ConnectionBase}}}}}, \sphinxcode{\sphinxupquote{object}}

\sphinxAtStartPar
Base class for a tcp connection.
\index{RECV\_MSGS\_POLLING\_INTERVAL (QConnectionLibrary.tcp.tcp\_base.TCPBase attribute)@\spxentry{RECV\_MSGS\_POLLING\_INTERVAL}\spxextra{QConnectionLibrary.tcp.tcp\_base.TCPBase attribute}}

\begin{fulllineitems}
\phantomsection\label{\detokenize{QConnectionLibrary:QConnectionLibrary.tcp.tcp_base.TCPBase.RECV_MSGS_POLLING_INTERVAL}}\pysigline{\sphinxbfcode{\sphinxupquote{RECV\_MSGS\_POLLING\_INTERVAL}}\sphinxbfcode{\sphinxupquote{ = 0.005}}}
\end{fulllineitems}

\index{address (QConnectionLibrary.tcp.tcp\_base.TCPBase property)@\spxentry{address}\spxextra{QConnectionLibrary.tcp.tcp\_base.TCPBase property}}

\begin{fulllineitems}
\phantomsection\label{\detokenize{QConnectionLibrary:QConnectionLibrary.tcp.tcp_base.TCPBase.address}}\pysigline{\sphinxbfcode{\sphinxupquote{property }}\sphinxbfcode{\sphinxupquote{address}}}
\sphinxAtStartPar
Get connection address.
\begin{description}
\item[{Returns:}] \leavevmode
\sphinxAtStartPar
Connection address.

\end{description}

\end{fulllineitems}

\index{close() (QConnectionLibrary.tcp.tcp\_base.TCPBase method)@\spxentry{close()}\spxextra{QConnectionLibrary.tcp.tcp\_base.TCPBase method}}

\begin{fulllineitems}
\phantomsection\label{\detokenize{QConnectionLibrary:QConnectionLibrary.tcp.tcp_base.TCPBase.close}}\pysiglinewithargsret{\sphinxbfcode{\sphinxupquote{close}}}{}{}
\sphinxAtStartPar
Close connection.
\begin{description}
\item[{Returns:}] \leavevmode
\sphinxAtStartPar
None.

\end{description}

\end{fulllineitems}

\index{conn\_timeout (QConnectionLibrary.tcp.tcp\_base.TCPBase property)@\spxentry{conn\_timeout}\spxextra{QConnectionLibrary.tcp.tcp\_base.TCPBase property}}

\begin{fulllineitems}
\phantomsection\label{\detokenize{QConnectionLibrary:QConnectionLibrary.tcp.tcp_base.TCPBase.conn_timeout}}\pysigline{\sphinxbfcode{\sphinxupquote{property }}\sphinxbfcode{\sphinxupquote{conn\_timeout}}}
\sphinxAtStartPar
Get connection timeout.
\begin{description}
\item[{Returns:}] \leavevmode
\sphinxAtStartPar
Connection timeout.

\end{description}

\end{fulllineitems}

\index{connect() (QConnectionLibrary.tcp.tcp\_base.TCPBase method)@\spxentry{connect()}\spxextra{QConnectionLibrary.tcp.tcp\_base.TCPBase method}}

\begin{fulllineitems}
\phantomsection\label{\detokenize{QConnectionLibrary:QConnectionLibrary.tcp.tcp_base.TCPBase.connect}}\pysiglinewithargsret{\sphinxbfcode{\sphinxupquote{connect}}}{}{}
\sphinxAtStartPar
\textgreater{}\textgreater{} Should be override in derived class.
Establish the connection.
\begin{description}
\item[{Returns:}] \leavevmode
\sphinxAtStartPar
None.

\end{description}

\end{fulllineitems}

\index{disconnect() (QConnectionLibrary.tcp.tcp\_base.TCPBase method)@\spxentry{disconnect()}\spxextra{QConnectionLibrary.tcp.tcp\_base.TCPBase method}}

\begin{fulllineitems}
\phantomsection\label{\detokenize{QConnectionLibrary:QConnectionLibrary.tcp.tcp_base.TCPBase.disconnect}}\pysiglinewithargsret{\sphinxbfcode{\sphinxupquote{disconnect}}}{\emph{\DUrole{n}{device}}}{}
\sphinxAtStartPar
\textgreater{}\textgreater{} Should be override in derived class.
Disconnect the connection.
\begin{description}
\item[{Returns:}] \leavevmode
\sphinxAtStartPar
None.

\end{description}

\end{fulllineitems}

\index{is\_connected (QConnectionLibrary.tcp.tcp\_base.TCPBase property)@\spxentry{is\_connected}\spxextra{QConnectionLibrary.tcp.tcp\_base.TCPBase property}}

\begin{fulllineitems}
\phantomsection\label{\detokenize{QConnectionLibrary:QConnectionLibrary.tcp.tcp_base.TCPBase.is_connected}}\pysigline{\sphinxbfcode{\sphinxupquote{property }}\sphinxbfcode{\sphinxupquote{is\_connected}}}
\sphinxAtStartPar
Get connected state.
\begin{description}
\item[{Returns:}] \leavevmode
\sphinxAtStartPar
True if connection is connected.
False if connection is not connected.

\end{description}

\end{fulllineitems}

\index{port (QConnectionLibrary.tcp.tcp\_base.TCPBase property)@\spxentry{port}\spxextra{QConnectionLibrary.tcp.tcp\_base.TCPBase property}}

\begin{fulllineitems}
\phantomsection\label{\detokenize{QConnectionLibrary:QConnectionLibrary.tcp.tcp_base.TCPBase.port}}\pysigline{\sphinxbfcode{\sphinxupquote{property }}\sphinxbfcode{\sphinxupquote{port}}}
\sphinxAtStartPar
Get connection port.
\begin{description}
\item[{Returns:}] \leavevmode
\sphinxAtStartPar
Connection port.

\end{description}

\end{fulllineitems}

\index{quit() (QConnectionLibrary.tcp.tcp\_base.TCPBase method)@\spxentry{quit()}\spxextra{QConnectionLibrary.tcp.tcp\_base.TCPBase method}}

\begin{fulllineitems}
\phantomsection\label{\detokenize{QConnectionLibrary:QConnectionLibrary.tcp.tcp_base.TCPBase.quit}}\pysiglinewithargsret{\sphinxbfcode{\sphinxupquote{quit}}}{\emph{\DUrole{n}{is\_disconnect\_all}\DUrole{o}{=}\DUrole{default_value}{True}}}{}
\sphinxAtStartPar
Quit connection.
\begin{description}
\item[{Args:}] \leavevmode
\sphinxAtStartPar
is\_disconnect\_all: Determine if it’s necessary for disconnect all connection.

\item[{Returns:}] \leavevmode
\sphinxAtStartPar
None.

\end{description}

\end{fulllineitems}

\index{socket\_instance (QConnectionLibrary.tcp.tcp\_base.TCPBase property)@\spxentry{socket\_instance}\spxextra{QConnectionLibrary.tcp.tcp\_base.TCPBase property}}

\begin{fulllineitems}
\phantomsection\label{\detokenize{QConnectionLibrary:QConnectionLibrary.tcp.tcp_base.TCPBase.socket_instance}}\pysigline{\sphinxbfcode{\sphinxupquote{property }}\sphinxbfcode{\sphinxupquote{socket\_instance}}}
\sphinxAtStartPar
Get method of socket\_instance property.
\begin{description}
\item[{Returns:}] \leavevmode
\sphinxAtStartPar
Value of \_socket\_instance.

\end{description}

\end{fulllineitems}

\index{timeout (QConnectionLibrary.tcp.tcp\_base.TCPBase property)@\spxentry{timeout}\spxextra{QConnectionLibrary.tcp.tcp\_base.TCPBase property}}

\begin{fulllineitems}
\phantomsection\label{\detokenize{QConnectionLibrary:QConnectionLibrary.tcp.tcp_base.TCPBase.timeout}}\pysigline{\sphinxbfcode{\sphinxupquote{property }}\sphinxbfcode{\sphinxupquote{timeout}}}
\sphinxAtStartPar
Get connection timeout value.
\begin{description}
\item[{Returns:}] \leavevmode
\sphinxAtStartPar
Value of connection timeout.

\end{description}

\end{fulllineitems}


\end{fulllineitems}

\index{TCPBaseClient (class in QConnectionLibrary.tcp.tcp\_base)@\spxentry{TCPBaseClient}\spxextra{class in QConnectionLibrary.tcp.tcp\_base}}

\begin{fulllineitems}
\phantomsection\label{\detokenize{QConnectionLibrary:QConnectionLibrary.tcp.tcp_base.TCPBaseClient}}\pysigline{\sphinxbfcode{\sphinxupquote{class }}\sphinxcode{\sphinxupquote{QConnectionLibrary.tcp.tcp\_base.}}\sphinxbfcode{\sphinxupquote{TCPBaseClient}}}
\sphinxAtStartPar
Bases: \sphinxcode{\sphinxupquote{object}}

\sphinxAtStartPar
Base class for TCP client.
\index{connect() (QConnectionLibrary.tcp.tcp\_base.TCPBaseClient method)@\spxentry{connect()}\spxextra{QConnectionLibrary.tcp.tcp\_base.TCPBaseClient method}}

\begin{fulllineitems}
\phantomsection\label{\detokenize{QConnectionLibrary:QConnectionLibrary.tcp.tcp_base.TCPBaseClient.connect}}\pysiglinewithargsret{\sphinxbfcode{\sphinxupquote{connect}}}{}{}
\end{fulllineitems}


\end{fulllineitems}

\index{TCPBaseServer (class in QConnectionLibrary.tcp.tcp\_base)@\spxentry{TCPBaseServer}\spxextra{class in QConnectionLibrary.tcp.tcp\_base}}

\begin{fulllineitems}
\phantomsection\label{\detokenize{QConnectionLibrary:QConnectionLibrary.tcp.tcp_base.TCPBaseServer}}\pysigline{\sphinxbfcode{\sphinxupquote{class }}\sphinxcode{\sphinxupquote{QConnectionLibrary.tcp.tcp\_base.}}\sphinxbfcode{\sphinxupquote{TCPBaseServer}}}
\sphinxAtStartPar
Bases: \sphinxcode{\sphinxupquote{object}}

\sphinxAtStartPar
Base class for TCP server.
\index{accept\_connection() (QConnectionLibrary.tcp.tcp\_base.TCPBaseServer method)@\spxentry{accept\_connection()}\spxextra{QConnectionLibrary.tcp.tcp\_base.TCPBaseServer method}}

\begin{fulllineitems}
\phantomsection\label{\detokenize{QConnectionLibrary:QConnectionLibrary.tcp.tcp_base.TCPBaseServer.accept_connection}}\pysiglinewithargsret{\sphinxbfcode{\sphinxupquote{accept\_connection}}}{}{}
\sphinxAtStartPar
Wrapper method for handling accept action of TCP Server.
\begin{description}
\item[{Returns:}] \leavevmode
\sphinxAtStartPar
None.

\end{description}

\end{fulllineitems}


\end{fulllineitems}

\index{TCPConfig (class in QConnectionLibrary.tcp.tcp\_base)@\spxentry{TCPConfig}\spxextra{class in QConnectionLibrary.tcp.tcp\_base}}

\begin{fulllineitems}
\phantomsection\label{\detokenize{QConnectionLibrary:QConnectionLibrary.tcp.tcp_base.TCPConfig}}\pysiglinewithargsret{\sphinxbfcode{\sphinxupquote{class }}\sphinxcode{\sphinxupquote{QConnectionLibrary.tcp.tcp\_base.}}\sphinxbfcode{\sphinxupquote{TCPConfig}}}{\emph{\DUrole{o}{**}\DUrole{n}{dictionary}}}{}
\sphinxAtStartPar
Bases: \sphinxcode{\sphinxupquote{QConnectionLibrary.utils.DictToClass}}

\sphinxAtStartPar
Class to store configurations for TCP connection.
\index{address (QConnectionLibrary.tcp.tcp\_base.TCPConfig attribute)@\spxentry{address}\spxextra{QConnectionLibrary.tcp.tcp\_base.TCPConfig attribute}}

\begin{fulllineitems}
\phantomsection\label{\detokenize{QConnectionLibrary:QConnectionLibrary.tcp.tcp_base.TCPConfig.address}}\pysigline{\sphinxbfcode{\sphinxupquote{address}}\sphinxbfcode{\sphinxupquote{ = \textquotesingle{}localhost\textquotesingle{}}}}
\end{fulllineitems}

\index{port (QConnectionLibrary.tcp.tcp\_base.TCPConfig attribute)@\spxentry{port}\spxextra{QConnectionLibrary.tcp.tcp\_base.TCPConfig attribute}}

\begin{fulllineitems}
\phantomsection\label{\detokenize{QConnectionLibrary:QConnectionLibrary.tcp.tcp_base.TCPConfig.port}}\pysigline{\sphinxbfcode{\sphinxupquote{port}}\sphinxbfcode{\sphinxupquote{ = 12345}}}
\end{fulllineitems}


\end{fulllineitems}

\phantomsection\label{\detokenize{QConnectionLibrary:module-QConnectionLibrary.tcp.ssh.ssh_client}}\index{module@\spxentry{module}!QConnectionLibrary.tcp.ssh.ssh\_client@\spxentry{QConnectionLibrary.tcp.ssh.ssh\_client}}\index{QConnectionLibrary.tcp.ssh.ssh\_client@\spxentry{QConnectionLibrary.tcp.ssh.ssh\_client}!module@\spxentry{module}}\index{AuthenticationType (class in QConnectionLibrary.tcp.ssh.ssh\_client)@\spxentry{AuthenticationType}\spxextra{class in QConnectionLibrary.tcp.ssh.ssh\_client}}

\begin{fulllineitems}
\phantomsection\label{\detokenize{QConnectionLibrary:QConnectionLibrary.tcp.ssh.ssh_client.AuthenticationType}}\pysigline{\sphinxbfcode{\sphinxupquote{class }}\sphinxcode{\sphinxupquote{QConnectionLibrary.tcp.ssh.ssh\_client.}}\sphinxbfcode{\sphinxupquote{AuthenticationType}}}
\sphinxAtStartPar
Bases: \sphinxcode{\sphinxupquote{object}}
\index{KEYFILE (QConnectionLibrary.tcp.ssh.ssh\_client.AuthenticationType attribute)@\spxentry{KEYFILE}\spxextra{QConnectionLibrary.tcp.ssh.ssh\_client.AuthenticationType attribute}}

\begin{fulllineitems}
\phantomsection\label{\detokenize{QConnectionLibrary:QConnectionLibrary.tcp.ssh.ssh_client.AuthenticationType.KEYFILE}}\pysigline{\sphinxbfcode{\sphinxupquote{KEYFILE}}\sphinxbfcode{\sphinxupquote{ = \textquotesingle{}keyfile\textquotesingle{}}}}
\end{fulllineitems}

\index{PASSWORD (QConnectionLibrary.tcp.ssh.ssh\_client.AuthenticationType attribute)@\spxentry{PASSWORD}\spxextra{QConnectionLibrary.tcp.ssh.ssh\_client.AuthenticationType attribute}}

\begin{fulllineitems}
\phantomsection\label{\detokenize{QConnectionLibrary:QConnectionLibrary.tcp.ssh.ssh_client.AuthenticationType.PASSWORD}}\pysigline{\sphinxbfcode{\sphinxupquote{PASSWORD}}\sphinxbfcode{\sphinxupquote{ = \textquotesingle{}password\textquotesingle{}}}}
\end{fulllineitems}

\index{PASSWORDKEYFILE (QConnectionLibrary.tcp.ssh.ssh\_client.AuthenticationType attribute)@\spxentry{PASSWORDKEYFILE}\spxextra{QConnectionLibrary.tcp.ssh.ssh\_client.AuthenticationType attribute}}

\begin{fulllineitems}
\phantomsection\label{\detokenize{QConnectionLibrary:QConnectionLibrary.tcp.ssh.ssh_client.AuthenticationType.PASSWORDKEYFILE}}\pysigline{\sphinxbfcode{\sphinxupquote{PASSWORDKEYFILE}}\sphinxbfcode{\sphinxupquote{ = \textquotesingle{}passwordkeyfile\textquotesingle{}}}}
\end{fulllineitems}


\end{fulllineitems}

\index{SSHClient (class in QConnectionLibrary.tcp.ssh.ssh\_client)@\spxentry{SSHClient}\spxextra{class in QConnectionLibrary.tcp.ssh.ssh\_client}}

\begin{fulllineitems}
\phantomsection\label{\detokenize{QConnectionLibrary:QConnectionLibrary.tcp.ssh.ssh_client.SSHClient}}\pysiglinewithargsret{\sphinxbfcode{\sphinxupquote{class }}\sphinxcode{\sphinxupquote{QConnectionLibrary.tcp.ssh.ssh\_client.}}\sphinxbfcode{\sphinxupquote{SSHClient}}}{\emph{\DUrole{o}{*}\DUrole{n}{args}}, \emph{\DUrole{o}{**}\DUrole{n}{kwargs}}}{}
\sphinxAtStartPar
Bases: {\hyperref[\detokenize{QConnectionLibrary:QConnectionLibrary.tcp.tcp_base.TCPBase}]{\sphinxcrossref{\sphinxcode{\sphinxupquote{QConnectionLibrary.tcp.tcp\_base.TCPBase}}}}}, {\hyperref[\detokenize{QConnectionLibrary:QConnectionLibrary.tcp.tcp_base.TCPBaseClient}]{\sphinxcrossref{\sphinxcode{\sphinxupquote{QConnectionLibrary.tcp.tcp\_base.TCPBaseClient}}}}}

\sphinxAtStartPar
SSH client connection class.
\index{close() (QConnectionLibrary.tcp.ssh.ssh\_client.SSHClient method)@\spxentry{close()}\spxextra{QConnectionLibrary.tcp.ssh.ssh\_client.SSHClient method}}

\begin{fulllineitems}
\phantomsection\label{\detokenize{QConnectionLibrary:QConnectionLibrary.tcp.ssh.ssh_client.SSHClient.close}}\pysiglinewithargsret{\sphinxbfcode{\sphinxupquote{close}}}{}{}
\sphinxAtStartPar
Close SSH connection.

\sphinxAtStartPar
Returns:

\end{fulllineitems}

\index{connect() (QConnectionLibrary.tcp.ssh.ssh\_client.SSHClient method)@\spxentry{connect()}\spxextra{QConnectionLibrary.tcp.ssh.ssh\_client.SSHClient method}}

\begin{fulllineitems}
\phantomsection\label{\detokenize{QConnectionLibrary:QConnectionLibrary.tcp.ssh.ssh_client.SSHClient.connect}}\pysiglinewithargsret{\sphinxbfcode{\sphinxupquote{connect}}}{}{}
\sphinxAtStartPar
Implementation for creating a SSH connection.
\begin{description}
\item[{Returns:}] \leavevmode
\sphinxAtStartPar
None.

\end{description}

\end{fulllineitems}

\index{quit() (QConnectionLibrary.tcp.ssh.ssh\_client.SSHClient method)@\spxentry{quit()}\spxextra{QConnectionLibrary.tcp.ssh.ssh\_client.SSHClient method}}

\begin{fulllineitems}
\phantomsection\label{\detokenize{QConnectionLibrary:QConnectionLibrary.tcp.ssh.ssh_client.SSHClient.quit}}\pysiglinewithargsret{\sphinxbfcode{\sphinxupquote{quit}}}{}{}
\sphinxAtStartPar
Quit and stop receiver thread.
\begin{description}
\item[{Returns:}] \leavevmode
\sphinxAtStartPar
None.

\end{description}

\end{fulllineitems}


\end{fulllineitems}

\index{SSHConfig (class in QConnectionLibrary.tcp.ssh.ssh\_client)@\spxentry{SSHConfig}\spxextra{class in QConnectionLibrary.tcp.ssh.ssh\_client}}

\begin{fulllineitems}
\phantomsection\label{\detokenize{QConnectionLibrary:QConnectionLibrary.tcp.ssh.ssh_client.SSHConfig}}\pysiglinewithargsret{\sphinxbfcode{\sphinxupquote{class }}\sphinxcode{\sphinxupquote{QConnectionLibrary.tcp.ssh.ssh\_client.}}\sphinxbfcode{\sphinxupquote{SSHConfig}}}{\emph{\DUrole{o}{**}\DUrole{n}{dictionary}}}{}
\sphinxAtStartPar
Bases: {\hyperref[\detokenize{QConnectionLibrary:QConnectionLibrary.tcp.tcp_base.TCPConfig}]{\sphinxcrossref{\sphinxcode{\sphinxupquote{QConnectionLibrary.tcp.tcp\_base.TCPConfig}}}}}

\sphinxAtStartPar
Class to store the configuration for SSH connection.
\index{address (QConnectionLibrary.tcp.ssh.ssh\_client.SSHConfig attribute)@\spxentry{address}\spxextra{QConnectionLibrary.tcp.ssh.ssh\_client.SSHConfig attribute}}

\begin{fulllineitems}
\phantomsection\label{\detokenize{QConnectionLibrary:QConnectionLibrary.tcp.ssh.ssh_client.SSHConfig.address}}\pysigline{\sphinxbfcode{\sphinxupquote{address}}\sphinxbfcode{\sphinxupquote{ = \textquotesingle{}localhost\textquotesingle{}}}}
\end{fulllineitems}

\index{authentication (QConnectionLibrary.tcp.ssh.ssh\_client.SSHConfig attribute)@\spxentry{authentication}\spxextra{QConnectionLibrary.tcp.ssh.ssh\_client.SSHConfig attribute}}

\begin{fulllineitems}
\phantomsection\label{\detokenize{QConnectionLibrary:QConnectionLibrary.tcp.ssh.ssh_client.SSHConfig.authentication}}\pysigline{\sphinxbfcode{\sphinxupquote{authentication}}\sphinxbfcode{\sphinxupquote{ = \textquotesingle{}password\textquotesingle{}}}}
\end{fulllineitems}

\index{key\_filename (QConnectionLibrary.tcp.ssh.ssh\_client.SSHConfig attribute)@\spxentry{key\_filename}\spxextra{QConnectionLibrary.tcp.ssh.ssh\_client.SSHConfig attribute}}

\begin{fulllineitems}
\phantomsection\label{\detokenize{QConnectionLibrary:QConnectionLibrary.tcp.ssh.ssh_client.SSHConfig.key_filename}}\pysigline{\sphinxbfcode{\sphinxupquote{key\_filename}}\sphinxbfcode{\sphinxupquote{ = None}}}
\end{fulllineitems}

\index{password (QConnectionLibrary.tcp.ssh.ssh\_client.SSHConfig attribute)@\spxentry{password}\spxextra{QConnectionLibrary.tcp.ssh.ssh\_client.SSHConfig attribute}}

\begin{fulllineitems}
\phantomsection\label{\detokenize{QConnectionLibrary:QConnectionLibrary.tcp.ssh.ssh_client.SSHConfig.password}}\pysigline{\sphinxbfcode{\sphinxupquote{password}}\sphinxbfcode{\sphinxupquote{ = \textquotesingle{}\textquotesingle{}}}}
\end{fulllineitems}

\index{port (QConnectionLibrary.tcp.ssh.ssh\_client.SSHConfig attribute)@\spxentry{port}\spxextra{QConnectionLibrary.tcp.ssh.ssh\_client.SSHConfig attribute}}

\begin{fulllineitems}
\phantomsection\label{\detokenize{QConnectionLibrary:QConnectionLibrary.tcp.ssh.ssh_client.SSHConfig.port}}\pysigline{\sphinxbfcode{\sphinxupquote{port}}\sphinxbfcode{\sphinxupquote{ = 22}}}
\end{fulllineitems}

\index{username (QConnectionLibrary.tcp.ssh.ssh\_client.SSHConfig attribute)@\spxentry{username}\spxextra{QConnectionLibrary.tcp.ssh.ssh\_client.SSHConfig attribute}}

\begin{fulllineitems}
\phantomsection\label{\detokenize{QConnectionLibrary:QConnectionLibrary.tcp.ssh.ssh_client.SSHConfig.username}}\pysigline{\sphinxbfcode{\sphinxupquote{username}}\sphinxbfcode{\sphinxupquote{ = \textquotesingle{}root\textquotesingle{}}}}
\end{fulllineitems}


\end{fulllineitems}

\phantomsection\label{\detokenize{QConnectionLibrary:module-QConnectionLibrary.tcp.raw.raw_tcp}}\index{module@\spxentry{module}!QConnectionLibrary.tcp.raw.raw\_tcp@\spxentry{QConnectionLibrary.tcp.raw.raw\_tcp}}\index{QConnectionLibrary.tcp.raw.raw\_tcp@\spxentry{QConnectionLibrary.tcp.raw.raw\_tcp}!module@\spxentry{module}}\index{RawTCPBase (class in QConnectionLibrary.tcp.raw.raw\_tcp)@\spxentry{RawTCPBase}\spxextra{class in QConnectionLibrary.tcp.raw.raw\_tcp}}

\begin{fulllineitems}
\phantomsection\label{\detokenize{QConnectionLibrary:QConnectionLibrary.tcp.raw.raw_tcp.RawTCPBase}}\pysiglinewithargsret{\sphinxbfcode{\sphinxupquote{class }}\sphinxcode{\sphinxupquote{QConnectionLibrary.tcp.raw.raw\_tcp.}}\sphinxbfcode{\sphinxupquote{RawTCPBase}}}{\emph{\DUrole{o}{*}\DUrole{n}{args}}, \emph{\DUrole{o}{**}\DUrole{n}{kwargs}}}{}
\sphinxAtStartPar
Bases: {\hyperref[\detokenize{QConnectionLibrary:QConnectionLibrary.tcp.tcp_base.TCPBase}]{\sphinxcrossref{\sphinxcode{\sphinxupquote{QConnectionLibrary.tcp.tcp\_base.TCPBase}}}}}

\sphinxAtStartPar
Base class for a raw tcp connection.

\end{fulllineitems}

\index{RawTCPClient (class in QConnectionLibrary.tcp.raw.raw\_tcp)@\spxentry{RawTCPClient}\spxextra{class in QConnectionLibrary.tcp.raw.raw\_tcp}}

\begin{fulllineitems}
\phantomsection\label{\detokenize{QConnectionLibrary:QConnectionLibrary.tcp.raw.raw_tcp.RawTCPClient}}\pysiglinewithargsret{\sphinxbfcode{\sphinxupquote{class }}\sphinxcode{\sphinxupquote{QConnectionLibrary.tcp.raw.raw\_tcp.}}\sphinxbfcode{\sphinxupquote{RawTCPClient}}}{\emph{\DUrole{o}{*}\DUrole{n}{args}}, \emph{\DUrole{o}{**}\DUrole{n}{kwargs}}}{}
\sphinxAtStartPar
Bases: {\hyperref[\detokenize{QConnectionLibrary:QConnectionLibrary.tcp.raw.raw_tcp.RawTCPBase}]{\sphinxcrossref{\sphinxcode{\sphinxupquote{QConnectionLibrary.tcp.raw.raw\_tcp.RawTCPBase}}}}}, {\hyperref[\detokenize{QConnectionLibrary:QConnectionLibrary.tcp.tcp_base.TCPBaseClient}]{\sphinxcrossref{\sphinxcode{\sphinxupquote{QConnectionLibrary.tcp.tcp\_base.TCPBaseClient}}}}}

\sphinxAtStartPar
Class for a raw tcp connection client.

\end{fulllineitems}

\index{RawTCPServer (class in QConnectionLibrary.tcp.raw.raw\_tcp)@\spxentry{RawTCPServer}\spxextra{class in QConnectionLibrary.tcp.raw.raw\_tcp}}

\begin{fulllineitems}
\phantomsection\label{\detokenize{QConnectionLibrary:QConnectionLibrary.tcp.raw.raw_tcp.RawTCPServer}}\pysiglinewithargsret{\sphinxbfcode{\sphinxupquote{class }}\sphinxcode{\sphinxupquote{QConnectionLibrary.tcp.raw.raw\_tcp.}}\sphinxbfcode{\sphinxupquote{RawTCPServer}}}{\emph{\DUrole{o}{*}\DUrole{n}{args}}, \emph{\DUrole{o}{**}\DUrole{n}{kwargs}}}{}
\sphinxAtStartPar
Bases: {\hyperref[\detokenize{QConnectionLibrary:QConnectionLibrary.tcp.raw.raw_tcp.RawTCPBase}]{\sphinxcrossref{\sphinxcode{\sphinxupquote{QConnectionLibrary.tcp.raw.raw\_tcp.RawTCPBase}}}}}, {\hyperref[\detokenize{QConnectionLibrary:QConnectionLibrary.tcp.tcp_base.TCPBaseServer}]{\sphinxcrossref{\sphinxcode{\sphinxupquote{QConnectionLibrary.tcp.tcp\_base.TCPBaseServer}}}}}

\sphinxAtStartPar
Class for a raw tcp connection server.

\end{fulllineitems}

\phantomsection\label{\detokenize{QConnectionLibrary:module-QConnectionLibrary.serialclient.serial_base}}\index{module@\spxentry{module}!QConnectionLibrary.serialclient.serial\_base@\spxentry{QConnectionLibrary.serialclient.serial\_base}}\index{QConnectionLibrary.serialclient.serial\_base@\spxentry{QConnectionLibrary.serialclient.serial\_base}!module@\spxentry{module}}\index{SerialClient (class in QConnectionLibrary.serialclient.serial\_base)@\spxentry{SerialClient}\spxextra{class in QConnectionLibrary.serialclient.serial\_base}}

\begin{fulllineitems}
\phantomsection\label{\detokenize{QConnectionLibrary:QConnectionLibrary.serialclient.serial_base.SerialClient}}\pysiglinewithargsret{\sphinxbfcode{\sphinxupquote{class }}\sphinxcode{\sphinxupquote{QConnectionLibrary.serialclient.serial\_base.}}\sphinxbfcode{\sphinxupquote{SerialClient}}}{\emph{\DUrole{o}{*}\DUrole{n}{args}}, \emph{\DUrole{o}{**}\DUrole{n}{kwargs}}}{}
\sphinxAtStartPar
Bases: {\hyperref[\detokenize{QConnectionLibrary:QConnectionLibrary.serialclient.serial_base.SerialSocket}]{\sphinxcrossref{\sphinxcode{\sphinxupquote{QConnectionLibrary.serialclient.serial\_base.SerialSocket}}}}}

\sphinxAtStartPar
Serial client class.
\index{connect() (QConnectionLibrary.serialclient.serial\_base.SerialClient method)@\spxentry{connect()}\spxextra{QConnectionLibrary.serialclient.serial\_base.SerialClient method}}

\begin{fulllineitems}
\phantomsection\label{\detokenize{QConnectionLibrary:QConnectionLibrary.serialclient.serial_base.SerialClient.connect}}\pysiglinewithargsret{\sphinxbfcode{\sphinxupquote{connect}}}{}{}
\sphinxAtStartPar
Connect to the Serial port.
\begin{description}
\item[{Returns:}] \leavevmode
\sphinxAtStartPar
None.

\end{description}

\end{fulllineitems}


\end{fulllineitems}

\index{SerialConfig (class in QConnectionLibrary.serialclient.serial\_base)@\spxentry{SerialConfig}\spxextra{class in QConnectionLibrary.serialclient.serial\_base}}

\begin{fulllineitems}
\phantomsection\label{\detokenize{QConnectionLibrary:QConnectionLibrary.serialclient.serial_base.SerialConfig}}\pysiglinewithargsret{\sphinxbfcode{\sphinxupquote{class }}\sphinxcode{\sphinxupquote{QConnectionLibrary.serialclient.serial\_base.}}\sphinxbfcode{\sphinxupquote{SerialConfig}}}{\emph{\DUrole{o}{**}\DUrole{n}{dictionary}}}{}
\sphinxAtStartPar
Bases: \sphinxcode{\sphinxupquote{QConnectionLibrary.utils.DictToClass}}

\sphinxAtStartPar
Class to store the configuration for Serial connection.
\index{baudrate (QConnectionLibrary.serialclient.serial\_base.SerialConfig attribute)@\spxentry{baudrate}\spxextra{QConnectionLibrary.serialclient.serial\_base.SerialConfig attribute}}

\begin{fulllineitems}
\phantomsection\label{\detokenize{QConnectionLibrary:QConnectionLibrary.serialclient.serial_base.SerialConfig.baudrate}}\pysigline{\sphinxbfcode{\sphinxupquote{baudrate}}\sphinxbfcode{\sphinxupquote{ = 115200}}}
\end{fulllineitems}

\index{bytesize (QConnectionLibrary.serialclient.serial\_base.SerialConfig attribute)@\spxentry{bytesize}\spxextra{QConnectionLibrary.serialclient.serial\_base.SerialConfig attribute}}

\begin{fulllineitems}
\phantomsection\label{\detokenize{QConnectionLibrary:QConnectionLibrary.serialclient.serial_base.SerialConfig.bytesize}}\pysigline{\sphinxbfcode{\sphinxupquote{bytesize}}\sphinxbfcode{\sphinxupquote{ = 8}}}
\end{fulllineitems}

\index{parity (QConnectionLibrary.serialclient.serial\_base.SerialConfig attribute)@\spxentry{parity}\spxextra{QConnectionLibrary.serialclient.serial\_base.SerialConfig attribute}}

\begin{fulllineitems}
\phantomsection\label{\detokenize{QConnectionLibrary:QConnectionLibrary.serialclient.serial_base.SerialConfig.parity}}\pysigline{\sphinxbfcode{\sphinxupquote{parity}}\sphinxbfcode{\sphinxupquote{ = \textquotesingle{}N\textquotesingle{}}}}
\end{fulllineitems}

\index{port (QConnectionLibrary.serialclient.serial\_base.SerialConfig attribute)@\spxentry{port}\spxextra{QConnectionLibrary.serialclient.serial\_base.SerialConfig attribute}}

\begin{fulllineitems}
\phantomsection\label{\detokenize{QConnectionLibrary:QConnectionLibrary.serialclient.serial_base.SerialConfig.port}}\pysigline{\sphinxbfcode{\sphinxupquote{port}}\sphinxbfcode{\sphinxupquote{ = \textquotesingle{}COM1\textquotesingle{}}}}
\end{fulllineitems}

\index{rtscts (QConnectionLibrary.serialclient.serial\_base.SerialConfig attribute)@\spxentry{rtscts}\spxextra{QConnectionLibrary.serialclient.serial\_base.SerialConfig attribute}}

\begin{fulllineitems}
\phantomsection\label{\detokenize{QConnectionLibrary:QConnectionLibrary.serialclient.serial_base.SerialConfig.rtscts}}\pysigline{\sphinxbfcode{\sphinxupquote{rtscts}}\sphinxbfcode{\sphinxupquote{ = False}}}
\end{fulllineitems}

\index{stopbits (QConnectionLibrary.serialclient.serial\_base.SerialConfig attribute)@\spxentry{stopbits}\spxextra{QConnectionLibrary.serialclient.serial\_base.SerialConfig attribute}}

\begin{fulllineitems}
\phantomsection\label{\detokenize{QConnectionLibrary:QConnectionLibrary.serialclient.serial_base.SerialConfig.stopbits}}\pysigline{\sphinxbfcode{\sphinxupquote{stopbits}}\sphinxbfcode{\sphinxupquote{ = 1}}}
\end{fulllineitems}

\index{xonxoff (QConnectionLibrary.serialclient.serial\_base.SerialConfig attribute)@\spxentry{xonxoff}\spxextra{QConnectionLibrary.serialclient.serial\_base.SerialConfig attribute}}

\begin{fulllineitems}
\phantomsection\label{\detokenize{QConnectionLibrary:QConnectionLibrary.serialclient.serial_base.SerialConfig.xonxoff}}\pysigline{\sphinxbfcode{\sphinxupquote{xonxoff}}\sphinxbfcode{\sphinxupquote{ = False}}}
\end{fulllineitems}


\end{fulllineitems}

\index{SerialSocket (class in QConnectionLibrary.serialclient.serial\_base)@\spxentry{SerialSocket}\spxextra{class in QConnectionLibrary.serialclient.serial\_base}}

\begin{fulllineitems}
\phantomsection\label{\detokenize{QConnectionLibrary:QConnectionLibrary.serialclient.serial_base.SerialSocket}}\pysiglinewithargsret{\sphinxbfcode{\sphinxupquote{class }}\sphinxcode{\sphinxupquote{QConnectionLibrary.serialclient.serial\_base.}}\sphinxbfcode{\sphinxupquote{SerialSocket}}}{\emph{\DUrole{o}{*}\DUrole{n}{args}}, \emph{\DUrole{o}{**}\DUrole{n}{kwargs}}}{}
\sphinxAtStartPar
Bases: {\hyperref[\detokenize{QConnectionLibrary:QConnectionLibrary.connection_base.ConnectionBase}]{\sphinxcrossref{\sphinxcode{\sphinxupquote{QConnectionLibrary.connection\_base.ConnectionBase}}}}}

\sphinxAtStartPar
Class for handling serial connection.
\index{connect() (QConnectionLibrary.serialclient.serial\_base.SerialSocket method)@\spxentry{connect()}\spxextra{QConnectionLibrary.serialclient.serial\_base.SerialSocket method}}

\begin{fulllineitems}
\phantomsection\label{\detokenize{QConnectionLibrary:QConnectionLibrary.serialclient.serial_base.SerialSocket.connect}}\pysiglinewithargsret{\sphinxbfcode{\sphinxupquote{connect}}}{}{}
\sphinxAtStartPar
Connect to serial port.
\begin{description}
\item[{Returns:}] \leavevmode
\sphinxAtStartPar
None.

\end{description}

\end{fulllineitems}

\index{disconnect() (QConnectionLibrary.serialclient.serial\_base.SerialSocket method)@\spxentry{disconnect()}\spxextra{QConnectionLibrary.serialclient.serial\_base.SerialSocket method}}

\begin{fulllineitems}
\phantomsection\label{\detokenize{QConnectionLibrary:QConnectionLibrary.serialclient.serial_base.SerialSocket.disconnect}}\pysiglinewithargsret{\sphinxbfcode{\sphinxupquote{disconnect}}}{\emph{\DUrole{n}{\_device}}}{}
\sphinxAtStartPar
Disconnect serial port.
\begin{description}
\item[{Args:}] \leavevmode
\sphinxAtStartPar
\_device: unused.

\item[{Returns:}] \leavevmode
\sphinxAtStartPar
None.

\end{description}

\end{fulllineitems}

\index{quit() (QConnectionLibrary.serialclient.serial\_base.SerialSocket method)@\spxentry{quit()}\spxextra{QConnectionLibrary.serialclient.serial\_base.SerialSocket method}}

\begin{fulllineitems}
\phantomsection\label{\detokenize{QConnectionLibrary:QConnectionLibrary.serialclient.serial_base.SerialSocket.quit}}\pysiglinewithargsret{\sphinxbfcode{\sphinxupquote{quit}}}{}{}
\sphinxAtStartPar
Quit serial connection.
\begin{description}
\item[{Returns:}] \leavevmode
\sphinxAtStartPar
None

\end{description}

\end{fulllineitems}


\end{fulllineitems}



\chapter{Indices and tables}
\label{\detokenize{index:indices-and-tables}}\begin{itemize}
\item {} 
\sphinxAtStartPar
\DUrole{xref,std,std-ref}{genindex}

\item {} 
\sphinxAtStartPar
\DUrole{xref,std,std-ref}{modindex}

\item {} 
\sphinxAtStartPar
\DUrole{xref,std,std-ref}{search}

\end{itemize}


\renewcommand{\indexname}{Python Module Index}
\begin{sphinxtheindex}
\let\bigletter\sphinxstyleindexlettergroup
\bigletter{q}
\item\relax\sphinxstyleindexentry{QConnectionLibrary.connection\_base}\sphinxstyleindexpageref{QConnectionLibrary:\detokenize{module-QConnectionLibrary.connection_base}}
\item\relax\sphinxstyleindexentry{QConnectionLibrary.connection\_manager}\sphinxstyleindexpageref{QConnectionLibrary:\detokenize{module-QConnectionLibrary.connection_manager}}
\item\relax\sphinxstyleindexentry{QConnectionLibrary.qlogger}\sphinxstyleindexpageref{QConnectionLibrary:\detokenize{module-QConnectionLibrary.qlogger}}
\item\relax\sphinxstyleindexentry{QConnectionLibrary.serialclient.serial\_base}\sphinxstyleindexpageref{QConnectionLibrary:\detokenize{module-QConnectionLibrary.serialclient.serial_base}}
\item\relax\sphinxstyleindexentry{QConnectionLibrary.tcp.raw.raw\_tcp}\sphinxstyleindexpageref{QConnectionLibrary:\detokenize{module-QConnectionLibrary.tcp.raw.raw_tcp}}
\item\relax\sphinxstyleindexentry{QConnectionLibrary.tcp.ssh.ssh\_client}\sphinxstyleindexpageref{QConnectionLibrary:\detokenize{module-QConnectionLibrary.tcp.ssh.ssh_client}}
\item\relax\sphinxstyleindexentry{QConnectionLibrary.tcp.tcp\_base}\sphinxstyleindexpageref{QConnectionLibrary:\detokenize{module-QConnectionLibrary.tcp.tcp_base}}
\end{sphinxtheindex}

\renewcommand{\indexname}{Index}
\printindex
\end{document}